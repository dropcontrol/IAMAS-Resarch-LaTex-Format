% IAMAS卒論Latexフォーマット ver.1.0

%uplatexコマンドでコンパイルする場合。UTF-8での欧文特殊文字が通るようになる
\documentclass[uplatex,dvipdfmx]{ujarticle}
%platexコマンドでコンパイルする場合は、
%上記をコメントアウトし、以下を使用すること。
%\documentclass[dvipdfmx]{jsarticle}

\usepackage[top=25truemm,bottom=30truemm,left=40truemm,right=40truemm]{geometry}

\usepackage{iamas}
\usepackage{amsmath}
\usepackage[dvipdfmx]{graphicx}
\usepackage{plext}
\usepackage{ascmac}
\usepackage{url}
\usepackage{wrapfig}
\usepackage{setspace}
\usepackage{chicago}
\usepackage{afterpage}
\usepackage{here}
\usepackage{caption}

%多言語対応用パッケージ。
%今の所utf8がないとか\u8がないとかエラーが出るが、強行して良い
%platexコマンドを使いたい場合は、ウムラウトやアクセント記号がついた文字
%をLaTeX記法(\"uなど)で置き換えた上で、以下のパッケージをコメントアウト
%すること
\usepackage{otf}
\usepackage[T1]{fontenc}
\usepackage[utf8]{inputenc}
\usepackage[prefernoncjk]{pxcjkcat}



\begin{document}
\large
% 表紙
\title{平成27年度 卒業論文\\
ポラーノの広場}
\author{宮沢賢治}
\date{2016年 3月}
\maketitle

%改ページや図版を章に収めたい場合は以下のように改ページすると収まる。
\afterpage{\clearpage}
\newpage

%アブストラクト
% Please add the following required packages to your document preamble:
% \usepackage{graphicx}
\begin{abstract}
\begin{table}[H]
\centering
\captionsetup{labelformat=empty,labelsep=none}
\caption{修士論文要旨}
\resizebox{\textwidth}{!}{
\large
\begin{tabular}{|l|l|l|l|l|l|l|l|l|l|l|}
\hline
\multicolumn{11}{|l|}{情報科学芸術大学院大学メディア表現研究科メディア表現専攻} \\ \hline
\multicolumn{2}{|l|}{修士論文提出者} & \multicolumn{2}{l|}{学籍番号} & \multicolumn{2}{l|}{11111} & \multicolumn{2}{l|}{名前} & \multicolumn{3}{l|}{いあます太郎} \\ \hline
\multicolumn{2}{|l|}{修士論文題名} & \multicolumn{9}{p{24em}|}{わたしのメディア表現学宣言 -機械とわたしの未来- 改行は24emで行なわれますので調整してください。} \\ \hline
\multicolumn{11}{|p{\textwidth}|}{一つ確実なのは、白い子ネコはなんの関係もなかったということ:――もうなにもかも、黒い子ネコのせいだったのです。というのも、白い子ネコは年寄りネコに、もう四半時も顔を洗ってもらっていたからです(そしてその状況を考えれば、なかなかがんばって耐えていたと言えましょう)。というわけで、白い子ネコはどう考えてもいたずらにはまったく荷担していなかったのはわかるでしょう。 ダイナはこんなふうにして子どもたちの顔を洗ったのでした:まずかわいそうな子を耳のところで前足片方を使っておさえこみ、そして残った前足で、子どもの顔中をこすります。それも鼻からはじめて変な方向に。そしてちょうどいま、ぼくがこうして話している間にも、ダイナはいっしょうけんめい白い子ネコを片づけています。白い子ネコはほとんど身動きせずに、のどをならそうとしていました――これもみんな自分のためを思ってのことなんだ、というのを感じていたのはまちがいありません。でも黒い子ネコは、午後の早い時期に顔を洗ってもらったので、アリスが半分ぶつぶつ、半分眠りながら、大きなソファのすみに丸まっている間に、アリスが巻いておこうとした毛糸の玉とせいだいにじゃれて、あちこちころがしてまわり、やがて毛糸玉はぜんぶほどけてしまいました。おかげで毛糸玉はこの通り、暖炉前のじゅうたんいちめんに広がって、そこらじゅうに結び目ができたりからまったりして、そのまん中で子ネコが自分のしっぽを追いかけているのでした。,「まあこのいたずらっ子め!」とアリスはさけんで子ネコを抱え上げ、ちょっとキスをして、しかられているんだとわからせてあげました。「まったく、ダイナがもっとちゃんとしつけてくれないと! そうでしょ、ダイナ、わかってるわよね!」とアリスはつけくわえながら、非難がましい目つきで年寄りネコのほうをながめて、できるだけきびしい声を出そうとします――それから子ネコと毛糸を持ってソファにかけもどり、また毛糸を巻きはじめました。でも、あまり手早くはありません。というのもときには子ネコに向かって、ときには自分に向かって、ずっとしゃべりどおしだったからです。子ネコちゃんはとてもとりすましてアリスのひざにすわり、毛糸を巻くすすみ具合を見ているふりをしつつ、ときどき前足を片方出して毛糸玉に軽くさわり、できるものなら喜んでお手伝いするところですが、とでも言うようです。(995文字)} \\ \hline
\multicolumn{2}{|l|}{論文審査員} & 主査 & \multicolumn{2}{l|}{三輪 眞弘} & 副査 & \multicolumn{2}{l|}{前田 真二朗} & 副査 & \multicolumn{2}{l|}{小林 昌弘} \\ \hline
\end{tabular}%
}
\end{table}
\setcounter{table}{0}

\begin{table}[H]
\centering
\captionsetup{labelformat=empty,labelsep=none}
\caption{ABSTRACT}
\resizebox{\textwidth}{!}{
\large
\begin{tabular}{|l|l|l|l|l|l|l|l|l|l|l|}
\hline
\multicolumn{11}{|p{\textwidth}|}{Institute of Advanced Media Arts and Sciences, The Graduate School of Media Creations, Course for Media Creations} \\ \hline
\multicolumn{2}{|l|}{Submitter} & \multicolumn{2}{l|}{Student ID} & \multicolumn{2}{l|}{11111} & \multicolumn{2}{l|}{Name} & \multicolumn{3}{l|}{TARO Iamas} \\ \hline
\multicolumn{2}{|l|}{Title} & \multicolumn{9}{p{24em}|}{わたしのメディア表現学宣言 -機械とわたしの未来- 改行は24emで行なわれますので調整してください。} \\ \hline
\multicolumn{11}{|p{\textwidth}|}{YOU don’t know about me without you have read a book by the name of The Adventures of Tom Sawyer; but that ain’t no matter. That book was made by Mr. Mark Twain, and he told the truth, mainly. There was things which he stretched, but mainly he told the truth. That is nothing. I never seen anybody but lied one time or another, without it was Aunt Polly, or the widow, or maybe Mary. Aunt Polly – Tom’s Aunt Polly, she is – and Mary, and the Widow Douglas is all told about in that book, which is mostly a true book, with some stretchers, as I said before.
Now the way that the book winds up is this: Tom and me found the money that the robbers hid in the cave, and it made us rich. We got six thousand dollars apiece – all gold. It was an awful sight of money when it was piled up. Well, Judge Thatcher he took it and put it out at interest, and it fetched us a dollar a day apiece all the year round – more than a body could tell what to do with. The Widow Douglas she took me for her son, and allowed she would sivilize me; but it was rough living in the house all the time, considering how dismal regular and decent the widow was in all her ways; and so when I couldn’t stand it no longer I lit out. I got into my old rags and my sugar-hogshead again, and was free and satisfied. But Tom Sawyer he hunted me up and said he was going to start a band of robbers, and I might join if I would go back to the widow and be respectable. So I went back.
The widow she cried over me, and called me a poor lost lamb, and she called me a lot of other names, too, but she never meant no harm by it. She put me in them new clothes again, and I couldn’t do nothing but sweat and sweat, and feel all cramped up. Well, then, the old thing commenced again. The widow rung a bell for supper, and you had to come to time. When you got to the table you couldn’t go right to eating, but you had to wait for the widow to tuck down her head and grumble a little over the victuals, though there warn’t really anything the matter with them, – that is, nothing only everything was cooked by itself. In a barrel of odds and ends it is different; things get mixed up, and the juice kind of swaps around, and the things go better.(452ワード)} \\ \cline{1-4}
\multicolumn{4}{|l|}{Examination Committee} & \multicolumn{7}{l|}{} \\ \hline
\multicolumn{4}{|l|}{Chief Examiner} & \multicolumn{7}{l|}{Masahiro MIWA} \\ \hline
\multicolumn{4}{|l|}{Co - Examiner} & \multicolumn{7}{l|}{Shinjiro MAEDA} \\ \hline
 \multicolumn{4}{|l|}{Co  Examiner} & \multicolumn{7}{l|}{Masahiro KOBAYASHI} \\ \hline
\end{tabular}%
}
\end{table}

\end{abstract}

%改ページや図版を章に収めたい場合は以下のように改ページすると収まる。
\afterpage{\clearpage}
\newpage

%目次
\tableofcontents

%改ページや図版を章に収めたい場合は以下のように改ページすると収まる。
\afterpage{\clearpage}
\newpage

% 本文
\section{背景、問題点など}
一つ確実なのは、白い子ネコはなんの関係もなかったということ:――もうなにもかも、黒い子ネコのせいだったのです。というのも、白い子ネコは年寄りネコに、もう四半時も顔を洗ってもらっていたからです(そしてその状況を考えれば、なかなかがんばって耐えていたと言えましょう)。というわけで、白い子ネコはどう考えてもいたずらにはまったく荷担していなかったのはわかるでしょう。 ダイナはこんなふうにして子どもたちの顔を洗ったのでした:まずかわいそうな子を耳のところで前足片方を使っておさえこみ、そして残った前足で、子どもの顔中をこすります。それも鼻からはじめて変な方向に。そしてちょうどいま、ぼくがこうして話している間にも、ダイナはいっしょうけんめい白い子ネコを片づけています。白い子ネコはほとんど身動きせずに、のどをならそうとしていました――これもみんな自分のためを思ってのことなんだ、というのを感じていたのはまちがいありません。でも黒い子ネコは、午後の早い時期に顔を洗ってもらったので、アリスが半分ぶつぶつ、半分眠りながら、大きなソファのすみに丸まっている間に、アリスが巻いておこうとした毛糸の玉とせいだいにじゃれて、あちこちころがしてまわり、やがて毛糸玉はぜんぶほどけてしまいました。おかげで毛糸玉はこの通り、暖炉前のじゅうたんいちめんに広がって、そこらじゅうに結び目ができたりからまったりして、そのまん中で子ネコが自分のしっぽを追いかけているのでした。,「まあこのいたずらっ子め!」とアリスはさけんで子ネコを抱え上げ、ちょっとキスをして、しかられているんだとわからせてあげました。「まったく、ダイナがもっとちゃんとしつけてくれないと! そうでしょ、ダイナ、わかってるわよね!」とアリスはつけくわえながら、非難がましい目つきで年寄りネコのほうをながめて、できるだけきびしい声を出そうとします――それから子ネコと毛糸を持ってソファにかけもどり、また毛糸を巻きはじめました。でも、あまり手早くはありません。というのもときには子ネコに向かって、ときには自分に向かって、ずっとしゃべりどおしだったからです。子ネコちゃんはとてもとりすましてアリスのひざにすわり、毛糸を巻くすすみ具合を見ているふりをしつつ、ときどき前足を片方出して毛糸玉に軽くさわり、できるものなら喜んでお手伝いするところですが、とでも言うようです。(995文字)

\subsection{サブセクション}
あのイーハトーヴォのすきとおった風、夏でも底に冷たさをもつ青いそら、うつくしい森で飾られたモリーオ市、郊外のぎらぎらひかる草の波。

\subsubsection{サブサブセクション}
あのイーハトーヴォのすきとおった風、夏でも底に冷たさをもつ青いそら、うつくしい森で飾られたモリーオ市、郊外のぎらぎらひかる草の波。

\section{目的、コンセプト、新規性など}

\section{過程、方法、手順など}

\section{調査、分析、実験、制作など}

\section{結果、成果など}

\section{考察、評価、展開など}

\end{document}
