% IAMAS卒論Latexフォーマット ver.1.0

%uplatexコマンドでコンパイルする場合。UTF-8での欧文特殊文字が通るようになる
\documentclass[uplatex,dvipdfmx]{ujarticle}
%platexコマンドでコンパイルする場合は、
%上記をコメントアウトし、以下を使用すること。
%\documentclass[dvipdfmx]{jsarticle}

\usepackage[top=25truemm,bottom=30truemm,left=30truemm,right=30truemm]{geometry}

\usepackage{iamas}
\usepackage{amsmath}
\usepackage[dvipdfmx]{graphicx}
\usepackage{plext}
\usepackage{ascmac}
\usepackage{url}
\usepackage{wrapfig}
\usepackage{setspace}
\usepackage{chicago}
\usepackage{afterpage}
\usepackage{here}
\usepackage{caption}

%多言語対応用パッケージ。
%今の所utf8がないとか\u8がないとかエラーが出るが、強行して良い
%platexコマンドを使いたい場合は、ウムラウトやアクセント記号がついた文字
%をLaTeX記法(\"uなど)で置き換えた上で、以下のパッケージをコメントアウト
%すること
\usepackage{otf}
\usepackage[T1]{fontenc}
\usepackage[utf8]{inputenc}
\usepackage[prefernoncjk]{pxcjkcat}

% 表紙
\title{\Large 多層時間構造による音楽創造の試みとその身体化の可能性 \\
Attempt to create music with multilayered temporal structures and possibility of embodiments for player and audience \\}
\author{大和 比呂志}
\date{\today}


\begin{document}

\maketitle
\large

%
% %改ページや図版を章に収めたい場合は以下のように改ページすると収まる。
% \afterpage{\clearpage}
% \newpage

%アブストラクト
% Please add the following required packages to your document preamble:
% \usepackage{graphicx}

% \begin{table}[H]
% \centering
% \captionsetup{labelformat=empty,labelsep=none}
% \caption{修士論文要旨}
% \resizebox{\textwidth}{!}{
% \large
% \begin{tabular}{|l|l|l|l|l|l|l|l|l|l|l|}
% \hline
% \multicolumn{11}{|l|}{情報科学芸術大学院大学メディア表現研究科メディア表現専攻} \\ \hline
% \multicolumn{2}{|l|}{修士論文提出者} & \multicolumn{2}{l|}{学籍番号} & \multicolumn{2}{l|}{16117} & \multicolumn{2}{l|}{名前} & \multicolumn{3}{l|}{大和 比呂志} \\ \hline
% \multicolumn{2}{|l|}{修士論文題名} & \multicolumn{9}{l|}{多層時間構造による音楽創造の試みとその身体化の可能性} \\ \hline
% \multicolumn{11}{|p{\textwidth}|}{本研究は2つの内容を扱っている。一つは2つ以上のテンポを特定の構造で用いた音楽を創造する試みについて、もう一つはそのような音楽作品の聴衆や実演家にとっての実現可能性について扱う。
%
% その背景には、現代の「音楽のリズム」の多くは4拍子、3拍子といった共通の拍子が支配的であり、これは音楽理論やMIDI/DTMといった現在の楽曲制作環境が理論と環境の両面に於いて西洋音楽における“楽譜”の影響から離れていないからではないかと推測する。しかし21世紀初頭にアフリカ系アメリカ人ミュージシャンらによって、意図的にリズムに“揺らぎ”や“訛り”といった要素が持ち込まれた。この事は将来リズムという観点で音楽を捉えた時に一つの分節点になり、西洋音楽における脱調性、ジャズにおけるモードの登場などと並ぶ大きな出来事として捉えられると筆者は考える。
%
% そうした背景からこの研究ではリズムに着目している、なぜなら音楽のリズムは「楽譜では捉えきれない多様性」を持っており、今後の音楽の視聴体験を更新する可能性があると考えられるからである。また現代において音というメディウムを扱うにあたって、テクノロジーとの関係並びに身体との関係を無視することはできない。
%
% 本文では、まずジャズからポップスという音楽の流れを、クラシックから現代音楽への流れになぞらえ捉えなおす。その両面に着目することで現代の音楽の在りようを明確にし、そこではジャズやポピュラー音楽のルーツの一部を形成し、西洋音楽の文脈では拾い切れなかったポリリズムやクロスリズムに着目する。次に複合的なリズムを使ったリズムのバリエーションを考察することで、現在行なわれているリズム的なアプローチとの差異や新規性を示していく。その上で、その方法を体系化し再現可能な形で提示する。この様な試みや考察を通じて提案する音楽を「演奏する」「聴く」という人々の行為に還元できると考えている。
%
% 本研究では、リズムの多様性を考察し、コンピューター/アルゴリズムといった方法やメソッドから、次の時代の「音楽と身体」における可能性を探求するものである。} \\ \hline
% \multicolumn{2}{|l|}{論文審査員} & 主査 & \multicolumn{2}{l|}{三輪 眞弘} & 副査 & \multicolumn{2}{l|}{前田 真二朗} & 副査 & \multicolumn{2}{l|}{小林 昌弘} \\ \hline
% \end{tabular}%
% }
% \end{table}
% \setcounter{table}{0}
%
% \begin{table}[H]
% \centering
% \captionsetup{labelformat=empty,labelsep=none}
% \caption{ABSTRACT}
% \resizebox{\textwidth}{!}{
% \large
% \begin{tabular}{|l|l|l|l|l|l|l|l|l|l|l|}
% \hline
% \multicolumn{11}{|p{\textwidth}|}{Institute of Advanced Media Arts and Sciences, The Graduate School of Media Creations, Course for Media Creations} \\ \hline
% \multicolumn{2}{|l|}{Submitter} & \multicolumn{2}{l|}{Student ID} & \multicolumn{2}{l|}{16117} & \multicolumn{2}{l|}{Name} & \multicolumn{3}{l|}{Hiroshi YAMATO} \\ \hline
% \multicolumn{2}{|l|}{Title} & \multicolumn{9}{p{30em}|}{Attempt to create music with multilayered temporal structures and possibility of embodiments for player and audience} \\ \hline
% \multicolumn{11}{|p{\textwidth}|}{This research deals with two subjects. One is about the attempt to create music using two or more tempos in a specific structure. The other concerns the possibility of embodying music for audiences and performers. 
%
% The rhythm of contemporary popular music is dominated by common rhythms such as a four-four time or three-four time. I infer that this is because the current environment of music creation such like both music theory and DTM/MIDI is influenced by the music score system. But in the beginning of the 21st century, African-American musicians introduced intentional polyrhythmic sway and accent. This watershed in the history of rhythm is considered as important as the appearance of the mode system in jazz and atonality in modern music.
%
% This is the background that informs my research on rhythm. Rhythm has a diversity that cannot be captured by the current score system, and it has the possibility of updating how people experience music. And when we deal with the medium of sound, we cannot ignore the relationships between sound and technology, and sound and the physical body.
%
% In the body of paper, first I reinterpret the movement of music from jazz to popular music, and compare this in parallel to the movement from classical music to modern music. This approach that focuses on both movements can make clear the current state of music. In this approach, I focus polyrhythm and crossrhythm that has failed to be correctly understood in the context of Western music . Second, I consider variations of rhythm using polyrhythm and other complex rhythm in order to show how this is different and novel when compared with the currently performed rhythmic approach. Then, I show a possible way of how people can perform with my method of rhythm. I believe it is possible that this approach can contribute to music experience and lead to the creation of new acts of ''playing'' and ''listening''.
%
% This research concerns that diversity of rhythm, and is an attempt to discover new possibilities for the relationship between music and the physical body for the next generation.} \\ \cline{1-4}
% \multicolumn{4}{|l|}{Examination Committee} & \multicolumn{7}{l|}{} \\ \hline
% \multicolumn{4}{|l|}{Chief Examiner} & \multicolumn{7}{l|}{Masahiro MIWA} \\ \hline
% \multicolumn{4}{|l|}{Co - Examiner} & \multicolumn{7}{l|}{Shinjiro MAEDA} \\ \hline
%  \multicolumn{4}{|l|}{Co - Examiner} & \multicolumn{7}{l|}{Masahiro KOBAYASHI} \\ \hline
% \end{tabular}%
% }
% \end{table}

\begin{figure}[ht]
\centerline{
	\includegraphics[
 width=\textwidth]{./PDF/abstract_jp.pdf}
}
\end{figure}

\begin{figure}[ht]
\centerline{
	\includegraphics[
 width=\textwidth]{./PDF/abstract_en.pdf}
}
\end{figure}


%改ページや図版を章に収めたい場合は以下のように改ページすると収まる。
\afterpage{\clearpage}
\newpage


%目次
\tableofcontents

%改ページや図版を章に収めたい場合は以下のように改ページすると収まる。
\afterpage{\clearpage}
\newpage

% 本文
\section{序論}
ジャズを含むポピュラー音楽における20世紀の100年を考え、その時代を表わすときに「{\bf 20世紀は、アメリカ合衆国南部で生まれた音楽が、さまざまに形を変えながら世界中に伝播していった100年だった}」\cite{murai:01}と言えるだろう。その発展は、初期にシートミュージックから始まり、ラジオの発明、レコード及びマスメディアによって拡散されていったということは言うまでもないが、なぜその音楽のリズムの多くに4拍子、または3拍子という共通の拍子が支配的なのだろうか。筆者はその理由として、現代に於いてもクラシック音楽から継承された「作曲者と演奏者を結ぶプロトコルとしての楽譜」という音楽を記述する為のシステムが、バークリー・メソッド等の音楽理論や現代の音楽制作環境に対して強く影響を与えているからだと考えている。

その一方で、多くの現代音楽やフリージャズ、ノイズ・ミュージック、エレクトロニカといった既存のリズムのフォームから離れる試みも多く行なわれている。しかし、その多くは近代のポピュラー音楽で部分的に要素として含まれることはあってもリズムのフォームを変更または更新するものではなかった。

そこで「ポリリズム/クロスリズム性」「リズムにおける揺らぎや訛り」といった、21世紀前後から近代におけるジャズのリズムのアプローチに着目し、その方法を拡大することを20世紀の音楽に対しての批評的試みとし、その手法として「多層時間構造による音楽」を考案した。その実例としての作品''{\bf Music for multilayered temporal structures}''\cite{yamato:01}という作品郡を作曲し、また実際に実演可能かを問う為にジャズ・ミュージシャンによる実演を試み、それを記録した。

\subsection{本研究であつかう範囲 \label{research_target}}
本研究では、序論で述べた通り「今後のリズムのアプローチの可能性」から現在のポピュラー音楽のリズムのフォームや、記譜やDTMといった制作環境などに対しての批評的な視点を提示していく。なぜそのような批評的試みを必要とするか、ということについてはユヴァル・ノア・ハラリは『サピエンス全史』の中で次のように述べている。\\

{\bf 中世の文化が騎士道とキリスト教との折り合いをつけられなかったのとちょうど同じように、現代の世界は、自由と平等との折り合いをつけられずにいる。だが、これは欠陥ではない。 このような矛盾はあらゆる人間文化につきものの、不可分の要素なのだ。それどころか、それは文化の原動力であり、私たちの種の創造性と 活力の根源でもある。対立する二つの音が同時に演奏されたときに楽曲が嫌でも進展する場合があるのと同じで、思考や概念や価値観の不協和音が起こると、私たちは考え、再評価し、批判することを余儀なくされる。調和ばかりでは、 はっとさせられることがない。
緊張や対立、解決不能のジレンマがどの文化にとってもスパイスの役割を果たすとしたら、どの文化に属する人間も必ず、矛盾する信念を抱き、相容れない価値観に引き裂かれることになる。これはどの文化にとっても本質的な特徴なので、「認知的不協和」という呼び名さえついている。認知的不協和は人間の心の欠陥と考えられることが多い。だが、じつは必須の長所なのだ。矛盾する信念や価値観を持てなかったとしたら人類の文化を打ち立てて維持することはおそらく不可能 だっただろう。} \cite{harari:01}\\

ここでハラリが述べている「不協和音」というのはもちろん音楽上の不協和音のことではなく、比喩であることは明らかだ。だが、図\ref{fig:figure_08}\cite{murai:01}に示されるようにジャズの起源の一つであり今日のポピュラー音楽に大きな影響を与えたブルース音楽が生れる契機にもなった、クレオール文化の中のアフリカ音楽とイスラム起源のアイルランド音楽の衝突も、ハラリの言う「人類が文化を打ち立てる為の認知的不協和」であると言える。またそのような認知的不協和が今日まで様々な形で起り、ポピュラー音楽として文化を維持、発展させていると考えられるだろう。

\begin{figure}[ht]
\centerline{
	\includegraphics[
 width=0.5\textwidth]{./PDF/figure_08.pdf}
}
\caption{村井 康司 (『あなたの聴き方を変えるジャズ史』p17, 2017)\\
中東イスラム圏の音楽がアフリカ音楽、アイルランド音楽へと影響を与え経由し、アメリカ大陸南部で合流する、というブルース起源を示した図}
\label{fig:figure_08}
\end{figure}

また、ハラリの言う不協和から調和への動きを「リズムへのアプローチ」で行い、視聴者、演奏者、作曲者に再評価や批判、批評を促し、それらよって「リズムを使った認知的不協和」生み、音楽という文化を先に進めることに貢献することがこの研究の目的である。

そのために、本研究とその過程で制作された作品は基本的にポピュラー音楽のコンテキストの上で語られることになる。具体的にはポピュラー音楽の基盤となったバークリー・メソッドの発達と、その理論の世界的普及の理由の一つでもあるリードシートによる記譜を元にしたジャズを基盤とした作品制作を通じ、リズムへのアプローチの提示、記譜方法、実演可能性といったことを述べていく。


\subsection{現代のポピュラー音楽史と西洋音楽史の類似}

本研究ではいわゆる「ポピュラー音楽」を取り上げているが、その背景について述べる。まず、本項で「西洋音楽」「ポピュラー音楽」という二つの軸で論を進めていくにあたってそれぞれを次の様に定義したい。

\begin{description}
 \item[西洋音楽]\mbox{}\\
	    主に20世紀以前のクラシック音楽から現代音楽までを指す。また20世紀以降の音楽であってもその様式に則った音楽であればこれに含む。
 \item[ポピュラー音楽]\mbox{}\\
	    主に20世紀以後のアメリカの民族音楽としてのブルースやジャズを母体にした音楽を指す。商業音楽全般を含む。
\end{description}

このような区分けの根拠として、ピーター・ファン=デル=マーヴェは『ポピュラー音楽の基礎理論』で次のように述べている。「{\bf ブルースとしての第一号が出版された時点で、20世紀のポピュラー音楽の作曲者が手にし得る材料はすっかり出揃った。以来、ポピュラー音楽はさまざまな民族様式を取り込み、クラシック音楽から発想の手掛かりを引き出し、目の前にあった諸様式をあらゆる手法で組みあわせた。(中略) しかし根本的に言えば、起源が1900年代以前にさかのぼらないような要素は、何ひとつ作り出されていない}」\cite{merwe:01}。ただし同時にマーヴェは「{\bf『ポピュラー音楽』という用語はレッテルであって、中身ではない。(中略) ポピュラーであるということは、それが『人々に好まれる』という意味にしろ『大衆の』という意味にしろ、一時的な価値でしかない}」\cite{merwe:01}とも述べている。この事は上記で分けた二つの定義はあくまで歴史的な背景に立つ便宜上の区分であり、その価値を指しているものではないことを示している。

このような区分が可能であるとして、ここで重要なのは先のマーヴェの「起源が1900年代以前にさかのぼらないような要素は、何ひとつ作り出されていない」という指摘にある。このことは「ポピュラー音楽」がその中に「西洋音楽」のコンテキストを含むことの自明性を示すが、筆者は「西洋音楽」と「ポピュラー音楽」をその発展史に立って並列に並べることで、その類似性と共に「西洋音楽」が取りこぼしたものと「ポピュラー音楽」の現在を提示することを試みたい。

まず「西洋音楽」を所謂「クラシック音楽から現代音楽(ジョン・ケージ、スティーブ・ライヒといったアメリカン・ニュー・ミュージックを含む)」の流れで考えたときに、17世紀から18世紀のバロック期での調性の確立、19世紀でのロマン派、オペラからアリアといった大衆化、そして20世紀での12音技法、セリーなど調性の拡大から現代音楽の流れとして大きく捉える。また「ポピュラー音楽」をアメリカの民族音楽であるブルースとジャズへの流れ、またそこで生れた「ポピュラー音楽理論」であるところのバークリー・メソッドやモードジャズ、フリージャズといった様式の多様化と理論化、その後のポップスの大量生産とメディア流通や発展による大衆化、エレクトロニクスによる多様性の拡大、といったものを比較したときどちらにも「調性の確立、拡大、脱調性」といった流れを観ることができる。特に20世紀に入ってからのバークリー・メソッドによって広められたコードシステムとそのシステムの分析性の高さは、その登場以前とは制作の現場を大きく変え「{\bf バークリー・メソッドは『商業音楽』史の中の、おそらくこれから長く続く歴史の中の、記号化の最初のピークになる}」\cite{kikuchi-ootani:01}と言えるだろう。この事は「{\bf 記号化=商業化=音韻重視 }」\cite{kikuchi-ootani:01}とも言え、西洋音楽に於ける「調性の確立から大衆化」と言う流れに類似している。また「西洋音楽」於ける「調性の拡大から現代音楽」への動きを「ポピュラー音楽」に於ける「手法としてのモード、フリー、ノイズから、テクノロジー/エレクトロニクスの発達と多様性」と並列的すると「調性の拡大から脱調性」といった動きとして両者を捉えることが可能だと考える。このように両者を並列に捉えることで、「システムの確立」「システムの拡大」「脱システム」といった流れの多くを「調性」を中心に両者共に行っているとも言える。またその立場に立って現代の「ポピュラー音楽」を捉えることで、今後の音楽の可能性を考えることが可能になる。

\section{ポリリズム/クロスリズムとは}

音楽之友社の『新音楽辞典』(第40版、1996年発行)では「ポリリズム」を「{\bf 対照的なリズムが2声部以上同時に用いられている現象をいい、クロス・リズムとも呼ばれる}」\cite{asaka:01}と述べている。また、J.S.バッハの「シンフォニア No.9」の冒頭を例に「{\bf ポリフォニー固有の声部の独立性を助長するためのリズムの対称性とはいちおう区別される}」\cite{asaka:01}としているが、現在においてまで「ポリリズム/クロスリズム」の定義は限定的に行なわれていないことが伺える。また、依田・小野は『ポリリズムの類型における楽理的分析』の中で「{\bf たった3つの音楽事典にあたっただけでも,ポリリズムの定義は非常に多様・曖昧であることが分かる。リズムの研究は古くから和声などと比ベると蔑ろにされてきた}」\cite{yoda-ono:01}と指摘し、ポリリズムの類型を示すことを試みている。和声に対してリズムについての研究の比重が高くなかったであろうことはジャン=イヴ・ボスールの『現代音楽を読み解く88のキーワード』などにあたっても、現代音楽においてもその多くが音韻、音律、音響についての技法であることからも解る。ただし、依田・小野が行った類型は詳細な分析がなされている一方で現代のポピュラー音楽では一般的でない西洋音楽的用語によって分類されるものになっている。ここでは依田・小野の分析とは別に現代の「ポピュラー音楽」の状況に照しあわせ、定義の基盤となる「アフリカのポリリズム」に言及し、「西洋音楽」が取りこぼしたものとしての、また「ポピュラー音楽」の現在としての「ポリリズム/クロスリズム」を取り上げる。

ここでは、本論文でのポリリズムの定義をベースとなる「非西洋音楽」としての「アフリカ」のリズムから解説し、本稿での「ポリリズム/クロスリズムの定義」を行い、続けて「インド」「中東」「ラテン」の各ジャンルのリズムの「ポリリズム/クロスリズム性」を検証する。その上で「西洋音楽」「ポピュラー音楽」についても同様に検証し、最後にリズムに関しての「西洋音楽」「ポピュラー音楽」の現在について考察する。

\subsection{非西洋音楽の中でのポリリズム/クロスリズム}
\subsubsection{アフリカ}

菊地・大谷は「{\bf 12による1周期が、アフリカ音楽のポリリズムの原型}」\cite{kikuchi-ootani:01}であり「{\bf アフリカの基礎ボリリズムは、4拍子と3拍子がクロスして}」\cite{kikuchi-ootani:01}いると述べている。これは塚田が『アフリカ音楽の正体』の中で、ブランデルの垂直的ヘミオラについて述べた内容と「アフリカに混在するリズム型(標準リズム型)とビートの関係」からもその傾向があると言っていいだろう\cite{tsukada:01}。

ブランデルの垂直型ヘミオラについて塚田は次のように述べている。「{\bf 『日の入りの歌』(図\ref{fig:figure_14})の歌と手拍子の関係に着目してみよう。手拍子は一貫して8分の6拍子、つまり2分割で進行するため、旋律が4分の3拍子で進行する部分(1、2、5、7小節目)では同時に2分割と3分割のリズム型が組み合わされる}」\cite{tsukada:01}。これがブランデルの言う垂直型ヘミオラである。また塚田はこの垂直的ヘミオラを「{\bf アフリカのポリリズムのもっとも原初的な形態と言えるもの}」\cite{tsukada:01}と述べている。また「{\bf このリズム構造の原型を高度に発達させたのが、アフリカの複雑なポリリズムだと考えれば良い。この垂直型ヘミオラこそ、かつてジョーンズがそのポリリズム論にて繰返し強調したアフリカの特徴的なリズム構造であり、また「日の入りの歌」に見られる手拍子のリズムこそ、アガウの言う「曲の背景一貫して存在する8分の6拍子」のこと}」\cite{tsukada:01}と過去のポリリズム研究との関連を述べその妥当性を示している。

\begin{figure}[ht]
\centerline{
	\includegraphics[
 width=\textwidth]{./PDF/figure_14.pdf}
}
\caption{塚田 健一 (『アフリカ音楽の正体』p41の楽譜 1-5, 2016)\\
各1、3、5、7小節に2拍と3拍のブランデルの垂直ヘミオラの形が見てとれる。赤枠と赤字は筆者による追記。菊地・大谷による「3拍子と4拍子がクロス」する構造を見ることができる}
\label{fig:figure_14}
\end{figure}

先に示した菊地・大谷による「アフリカ音楽のポリリズムの原型」は、塚田の述べた垂直的ヘミオラで引用した図\ref{fig:figure_14}の中でも見ることができる。図\ref{fig:figure_14}の赤枠で囲んだ7、8小節目に注目し、その2小節を12拍としてを一つの纒まりとして見ることで、手拍子は4回均等に打たれていることがわかる。また歌の7小節目の1、3拍目の4分音符、8小節目の2拍目の8分音符をに注目すれば全体の中の均等な3つの打点を取ることができ、8分音符が歌は4+4+4、手拍子は3+3+3+3と12拍周期での同じ形を確認することができる。

また、12拍周期のリズム型については「{\bf アフリカ大陸の北からサハラ砂漠を超えると、突然そのリズム型がいたるところで聞こえてくる}」\cite{tsukada:01}とされ、その基礎リズムは図\ref{fig:figure_15}のようにまとめられている。また、その基礎リズムは図\ref{fig:figure_16}の様に「{\bf アフリカ人が標準リズム型を演奏するときには、8分音符を『基本パルス』として感じながら演奏する}」\cite{tsukada:01}とし、さらに「{\bf 標準リズム型を演奏する際に、この基本パルス3つをひとまとまり(ビート)として感じている}」\cite{tsukada:01}ところから、菊地・大谷の定義するような形で演奏がされてきたと考えられる。

\begin{figure}[ht]
\centerline{
	\includegraphics[
 width=0.5\textwidth]{./PDF/figure_15.pdf}
}
\caption{塚田 健一 (『アフリカ音楽の正体』p59の楽譜 2-4, 2016)}
\label{fig:figure_15}
\end{figure}

\begin{figure}[ht]
\centerline{
	\includegraphics[
 width=0.75\textwidth]{./PDF/figure_16.pdf}
}
\caption{塚田 健一 (『アフリカ音楽の正体』p61の楽譜 2-5, 2016)}
\label{fig:figure_16}
\end{figure}

\subsubsection{ポリリズム/クロスリズムの定義}

ここまで述べたように一般的な「ポリリズム」の解釈は広義なものであるが、アフリカのポリリズムを見ていくと一定の構造を見出すことができる。その構造について、菊地、大谷は「{\bf 3と4のクロスリズムはもはや、僕の定義ではポリリズムじゃない。あれはアフリカ式のモノリズムだ(中略)そこから生れる訛りだとか時間感覚の短縮とかに入って始めてアフリカンなポリリズミック・ステージに入る(中略)今の商業音楽のの中ではもう、さっきやった3と4のクロスは基本的なものだと考えてしまいたい}」\cite{kikuchi-ootani:01}と述べている。塚田も垂直的ヘミオラの商業音楽での出現の例として「{\bf 日本のテクノポップ・ユニット、Perfume(パフューム)が2007年にヒットさせた「ポリリズム」(中田ヤスタカ作曲)という曲では間奏部分に強烈で印象的な垂直的ヘミオラの構造が表われる」}\cite{tsukada:01}と指摘している。

そこでこの論文での「ポリリズム/クロスリズム」は菊地・大谷の定義を元に以下の様に定義する。

\begin{description}
 \item[クロスリズム]\mbox{}\\
	    2つの以上の拍子が、ある1周期単位を整数比で割きれるもの。垂直的ヘミオラ的リズム。
 \item[ポリリズム]\mbox{}\\
	    2つの以上の拍子が、ある1周期単位を整数比では割きれないもの。揺れや訛りを持つリズム。
\end{description}

これに従えば、前述のアフリカの基礎的なポリリズムとされていた12拍周期を3拍と4拍のビートを持つリズムはそれぞれ12が最小公倍数とされ、クロスリズムとされる。ポリリズムの状態とは、簡単な例で言えば、12周期に対して、4拍と5拍のビートを持つリズムであれば、12拍を4拍で割れば3拍単位のビートとなり整数で割り切れるが、5で割った場合は2.4拍単位のビートとなり整数で割りきれないものとなる。

以下この定義を元にインド、中東、ラテンのポリリズム/クロスリズムの状況を見ていくことにする。

\subsubsection{インド}

インド音楽では西洋音楽と「{\bf 同様に小節という形を取るが、時間分割の配列の種類を数多く作っていった}」\cite{deva:01}ターラと呼ばれるリズム形式を持っている。ターラは「{\bf あるパターンの周期性をもった配列と定義され}」\cite{deva:01}、「{\bf ターラの本質はその周期性、反復性}」\cite{deva:01}にある。ターラを用いたインド音楽でのポリリズム/クロスリズム性についてザックスは『リズムとテンポ』の中で「対位リズム」と「2つの型のオーヴァーラップ」という言葉で示している。

まず「対位リズム」については「{\bf 単純な対位リズムは、同じターラを用いる。つまり右手で標準テンポで型を奏し、左手でその長さを拡大する}」\cite{sachs:01}としている。図\ref{fig:figure_17}はその図示にあたる。

\begin{figure}[ht]
\centerline{
	\includegraphics[
 width=0.5\textwidth]{./PDF/figure_17.pdf}
}
\caption{クルト・ザックス (『リズムとテンポ』p106, 1979)\\
対立リズムの例。右手のリズムに対して左手は倍の長さに拡大されている}
\label{fig:figure_17}
\end{figure}

また別の「対位リズム」の例として「{\bf 両手が異なるターラを奏することがしばしばある。次のように、一方の手は標準テンポで、他方の手は拡大したテンポで奏する}」\cite{sachs:01}とし、図\ref{fig:figure_18}でそれを示す。

\begin{figure}[ht]
\centerline{
	\includegraphics[
 width=0.5\textwidth]{./PDF/figure_18.pdf}
}
\caption{クルト・ザックス (『リズムとテンポ』p107, 1979)\\
右手と左手で異るターラ演奏する例}
\label{fig:figure_18}
\end{figure}

「2つの型のオーヴァーラップ」の例としては次の図\ref{fig:figure_19}を示している。

\begin{figure}[ht]
\centerline{
	\includegraphics[
 width=0.7\textwidth]{./PDF/figure_19.pdf}
}
\caption{クルト・ザックス (『リズムとテンポ』p108, 1979)\\
2つの型のオーヴァーラップの例。アフリカの垂直ヘミオラと同じ構造が見てとれる}
\label{fig:figure_19}
\end{figure}

この形は先のポリリズムの定義であれば、20拍1周期の5拍と4拍のクロスリズムとなり、インド音楽においても同類のリズムを確認することが出来る。

% %改ページや図版を章に収めたい場合は以下のように改ページすると収まる。
% \afterpage{\clearpage}
% \newpage

\subsubsection{中東}

中東のリズムはサラーフ・アル・マハディの『アラブ音楽』によるとイーカーと呼ばれるリズムの型を持っていることが解る。マハディはアラブのリズムを「{\bf アラビア語を話す民族に特有なリズム体系}」\cite{mahdi:01}であり「{\bf もともと詩の韻律に結びついたもので(中略)それが次第に少しずつ独立したものとなった}」\cite{mahdi:01}と述べている。ザックスも「{\bf アラビア音楽は元来は詩のリズムに従っていた}」\cite{sachs:01}と述べている。中東の音楽(ザックスは中近東としてる)のポリリズム性についてザックスは「{\bf 音の三つの要素---強さ、音色、拍子---が同時に起ると、どの単一の型の内部でもほとんどポリリズム的な韻律とアクセントの複合した集合ができあがる。それは、長さにおいて、アクセントと音色の配分において、無限に変化しうるものであり、また変化してきた}」\cite{sachs:01}とし、これは先の定義から言えば広義な意味での「ポリリズム的」ではあってもインド音楽に見られたような明確なクロスリズムの形を類例を見付けることが今回は出来なかった。

% 図\ref{fig:figure_20}にあるチェニジアの例を上げている。

% \begin{figure}[ht]
% \centerline{
% 	\includegraphics[
%  width=\textwidth]{./PDF/figure_20.pdf}
% }
% \caption{クルト・ザックス (『リズムとテンポ』p88, 1979)\\
% 譜例は上段の歌に対して下段のリズムが特に周り混むこともなくアフリカやインドの例のような明確なクロスリズムの形を持っていない}
% \label{fig:figure_20}
% \end{figure}

\subsubsection{ラテン}

ラテンのリズムについては、中南米のパーカッションであるクラベスによって打ち出される「ラテン音楽の要となる」\cite{yubi:01}クラーベというリズムの型を取りあげる。塚田は「{\bf 興味深いことは、この標準リズム型が今日われわれのよく耳にするラテン音楽のリズムの基礎を形成したと考えられる}」\cite{tsukada:01}と述べている。またその類似を「{\bf アフロ・キューバ音楽の、たとえばルンバのリズムを詳細に知らべていくと、クラーベと標準リズム型との間には強い近親性がある}」\cite{tsukada:01}と指摘している。また「{\bf ルンバはアフリカのリズム構造の特徴をきわめて良く残していて、3拍系(8分の6拍子あるいは8分の12拍子)のリズムと2拍子系(4分の2拍子あるいは4分の4拍子)のリズムがパートが分れて同時に進行していく。つまり、ポリリズムの構造}」\cite{tsukada:01}であることも同時に指摘している。

図\ref{fig:figure_21}に、その類似とポリリズム構造を示す。この図から「{\bf 標準リズム型A(著者注・図の2-9の標準リズム型を指す)を本来の3拍子系ではなく、2拍子系のリズムで演奏しようとしたときに生じるのが、クラーベのリズム系}」\cite{tsukada:01}であることが伺える。

また、このクラーベから派生する「トレシージョ」(図\ref{fig:figure_21}の楽譜 2-11)と呼ばれるリズムは「{\bf そのヴァリエーションを含めると、ラテン音楽ばかりでなく、リズム・アンド・ブルース、ジャズ、ロックンロールなどにも多様され、その後の世界のポピューラー音楽の形成と発展に甚大は影響を及ぼした。そしてそのもととなったリズムが(中略)アフリカ伝統音楽の標準リズム型だった}」\cite{tsukada:01}ことからポピュラー音楽のアフリカ音楽起源を「{\bf 主観的な印象や根拠薄弱は憶測などではなく、ある意味で音楽分析上導き出された、歴史的事実に合致する言説}」\cite{tsukada:01}であり、本研究でのポピュラー音楽の発展を「ポリリズム/クロスリズムからアプローチする」という方法論の妥当性を音楽史的に認めることが出来ると筆者は考える。

\begin{figure}[ht]
\centerline{
	\includegraphics[
 width=0.75\textwidth]{./PDF/figure_21.pdf}
}
\caption{塚田 健一 (『アフリカ音楽の正体』p75 楽譜 2ー9, 楽譜 2-10, 楽譜2-11, 2016)\\
アフリカの標準リズム型をクラーベの類似性を示している図と、トレシージョのパターン}
\label{fig:figure_21}
\end{figure}

また、クラーベのもつ「ポリリズム性」と「アフリカの基礎リズム」との関係については、''Rumba Clave: An Illustrated Analysis''によって波形を用いて図示されたリズムの打点を検証することが出来る。ここでは、4分の4拍子系のクラーベの打点と、8分の12拍子系のクラーベの打点を明示した上で、10種類の奏者の打点を比較している。この比較から塚田の「クラーベのアフリカ起源」の跡が伺える。また、打点が4分の4拍子系と8分の12拍子系に揺れている状態を視覚的に見てとることができ、クロスリズムのような形は一見取っていないものの、クラーベがアフリカ起源のリズムであることと、このような「揺れ」「訛り」を持つリズムが、先のポリリズム定義の際に引用した菊地、大谷の「アフリカンなポリリズミック・ステージ」の一つの表われであり、このリズムの上で「{\bf アフリカ固有の標準リズム型の基本パルスの感じ方を身に付けていないスペイン系キューバ人が、標準リズム型を16パルスからなるリズム型として聴き取った}」\cite{tsukada:01}のであれば、前述の定義に於てもきわめてポリリズム的な音楽であると言える。

\subsection{西洋音楽の中でのポリリズム/クロスリズム}
\subsubsection{西洋音楽、現代音楽、アメリカン・ニュー・ミュージック \label{Western-Modern-AmericanNewMusic}}

西洋音楽に於ける「ポリリズム/クロスリズム」について、再び『新音楽辞典』(第40版、1996年発行)では「{\bf ポリリズム(メトリック)の用例は、ほぼ14-15世紀と20世紀に数多く見られるが、このような現象は音楽が調性機能および機能和声から自由であることと密接に関係している}」\cite{asaka:01}とある。図\ref{fig:figure_22}は、そこで図示されている譜例である。

\begin{figure}[ht]
\centerline{
	\includegraphics[
 width=0.75\textwidth]{./PDF/figure_22.pdf}
}
\caption{淺香 淳 (『新音楽辞典』p538 譜例 5 と 譜例 6, 1977)\\
拍子記号が段によって違うことで小節線が周り込んでいる例}
\label{fig:figure_22}
\end{figure}

バロック以前については、ザックスの『リズムとテンポ』でも、ルネサンスまでのポリリズムについては「韻律の交代」「フランボワイヤン・ポリリズム」、後期ルネサンス期のチプリアーノ・デ・ローレの「潔き処女」の例を上げている(図\ref{fig:figure_23})。

\begin{figure}[ht]
\centerline{
	\includegraphics[
 width=0.75\textwidth]{./PDF/figure_23.pdf}
}
\caption{クルト・ザックス (『リズムとテンポ』p272 Ex. 84 チプリアーノ・デ・ローレの「潔き処女」譜例, 1979)\\
図\ref{fig:figure_22}と同じく拍子記号が段によって違うことで小節線が周り込んでいるが周回的なリズムパターンのような形は見られない}
\label{fig:figure_23}
\end{figure}

バロック期に入ると「ポリリズム/クロスリズム」的なリズムは舞踏との関係の中でヘミオラ構造を見ることができるが、音楽のリズムの特徴として見ることはできない。またバロック期に4分の4拍子の指導的な地位が確立したが、それはその後の西洋音楽、ポピュラー音楽のリズムの基盤ともなっている。ライヒは「{\bf バロック音楽全般に言えることだが、特にバッハは、ビートが一定で基本的にリタルダンドやアッチェレランドがない(中略)するべきことはテンポを保つことだ}」\cite{strickland:01}と指摘しているところからも、バロック期のリズムがポリリズム的なものを持たなくなっていたことが解る。

ライヒは、20世紀に入り「ポリリズム/クロスリズム」的なリズムのアプローチをした現代音楽の作家の一人として思い起すことが出来る。この時取り上げられるライヒの代表的な手法がフェーズシフトだが、フェーズシフトに関してライヒはテープループ作品から着想を得たカノン形式のヴァリエーションであることをインタビューで語っている(「{\bf 何てことだ!この関係はすごい。けれど本当に面白いのは、ユニゾンで始まって、ゆっくりとフェーズがずれていくというプロセスであって、これは実際には古い西洋音楽のカノンのヴァリエーションだ!}」\cite{obrist:01})。ライヒは同インタビューにて「{\bf 私の最初のアンサンブルの結成メンバーとなった、アーサー・マーフィーと演奏してみると、なんと、私たちだけでもテープを使わずに弾くことが出来ることがわかりました。1967年のことでしたが、これが大きな突破口となって「ピアノ・フェイズ」「ヴァイオリン・フェイズ」(1967)、そしてその後の「ドラミング」(1970ー1971)が誕生}」\cite{obrist:01}したと語っており、一連のフェーズシフト作品はアフリカ的ポリリズムの「揺れ」「訛り」というよりは「ずれ」ていく時間的変化によるカノンであると言っている。たがライヒは自身がガムラン音楽を学んだ体験との関係性を問われた別のインタビューでは「{\bf (著者注・マレット楽器と声とオルガンための音楽ついて)確かに、録音はいろいろ聞いていた。別の例が「ドラミング」だ。ガーナへ行ったから書いたのではない。(中略)あの作品でのアフリカ的なもの---8分の12拍子の反復パターン---は1967年の「ピアノ・フェイズ」「ヴァイオリン・フェイズ」ですでに行っている。アフリカには確信するために行ったようなものだ}」\cite{strickland:01}と、アフリカの基本リズム構造の8分の12拍子の反復を使いつつも、それが先行作品からの影響でありアフリカ音楽そのものの影響であることを否定している。この為、ライヒの作品を先の定義の「ポリリズム/クロスリズム」に位置付けるのは作家自身の見解(インタビュー)からも困難であると考える。

\subsection{ポピュラー音楽の中でのポリリズム/クロスリズム}

ここまでアフリカ、インド、中東、ラテン、そして西洋音楽とその中で行なわれてきたリズムについて限定的にではあるが見てきた。音楽を総括的に語るためには、その音楽のすべての類例にあたらない限り不可能であるとも言える。さらにポピュラー音楽に関しては今現在、進行形の音楽のため、その出自をアメリカの民族音楽のジャズやブルースと限定しても総括的に語るという難易度が飛躍的にあがる。従って、本論文のポリリズム定義の基盤となる著述をした、菊地・大谷のうち菊地が行ってきた活動を紹介する。

菊地 成孔の活動を見たときに、彼が数多くのポリリズム的なアプローチによる実作と実演を行っている事が解る。ここではその一例として''Date Course Pentagon Royal Garden (現 dCprG)''を取り上げる。菊地は『官能と憂鬱を教えた学校』にて氏の楽曲である''PLAYMATE AT HANOI''\cite{works-kikuchi:01}について「{\bf 4と3のクロスです。(中略)整数的なポリ}」\cite{kikuchi-ootani:01}であると述べている。楽曲が進むにつれ「4と3が混った状態になる」とは言っているが、この言説の部分はこの楽曲がまず垂直的ヘミオラの状態であり、この論文での定義であれば「クロスリズム」の状態であることを指している。

そして、菊地の同バンドによる''structure I la structure de la magie monderne /構造 1 (現代呪術の構造)''\cite{works-kikuchi:02}では、先の''PLAYMATE AT HANOI''と比較してそのリズム構造について次のように説明している。「{\bf ここでの構造は4分の4と4分の5。つまり4拍子と5拍子が同居している。というのが特徴で、(中略)いわゆる前作の「プレイメイト・アット・ハノイ」のような、 12素数の割り切れるコスモス・ポリリズムではなく、片方から見るとスクエアなリズム(5拍子で取ると、16分音符を基調にしたハウスみたいに聴こえる)だが、 片方から見ると「訛って」聴こえる(4拍子で取ると、津軽民謡とか、アフリカ民謡みたいなつんのめったトライヴァリックが聴こえる)という物になっています。 ブラスのリフも、総て16分音符だけのヴァリエーションで出来ていますが、通常打たないような打点に打点を多く打っているので、 ステップの不穏さと快楽感が混じるような意図があります。更に最初に出てくる呪術としてのシンセサイザーの長いメロディーは、三連符で構成されており、 整数的近似値で言うと、三連符は4拍子と5拍子の間に位置し、ふたつのリズムの接着剤のような働きをしています。}」\cite{kikuchi:01}。この説明は菊地・大谷の定義を元にした本稿の「ポリリズム」の定義と一致する類例となる。

またこのような明快な例でなくとも、先にラテンについての節でアフリカの基礎リズムからラテンのクラーベへの繋りや、そこからポピュラー音楽に広く影響を与えていることもポピュラー音楽が常にアフリカ的ポリリズム感覚を受けていることを示している。また、20世紀初頭のポピュラー音楽であったジャズを見てみると西洋音楽とジャズを比較して「{\bf クラシックでは全パートが和声に支配されて互いに垂直的に連動するのに対して、ジャズでは和声に支配されながらも、各パートのリズムは線的もしくは水平的に独立して、しばしばポリリズムを形成する}」\cite{yubi:01}というところからもアフリカ的ポリリズム感覚の影響が伺えるだろう。また由比は「{\bf 1940年代に始まるバップbop(bebop)・ムーヴメントは、インプロビゼーションにおける表現度の拡大をめざすとともに、微小音価のアクセント移行を導入することによって、スイング感よりもより強力なポリリズムを形成}」\cite{yubi:01}したと指摘している。ビバップ以降のジャズ・ミュージシャンのアフリカ的ポリリズム感の例として、図\ref{fig:figure_24}に、ジョン・コルトレーンの''Bye Bye Blackbird''\cite{miles:03}でのソロの一部を示す。ここでコルトレーンは、4分の4拍子のリズムの上で2拍3連のフレーズを使い、4拍子上に6拍子を表現している。

\begin{figure}[ht]
\centerline{
	\includegraphics[
 width=\textwidth]{./PDF/figure_24.pdf}
}
\caption{Hal Leonard Corp (『John Coltrane Omnibook: For C Instruments』p50 ''Bye Bye Blackbird''でのソロの譜例より抜粋, 2013)\\
譜例3の2小節目で16分音符による4拍フレーズから入り、同じく2小節目4拍目のフレーズの形をモチーフにして3小節目ではそれを2拍3連とし、4拍を6分割している}
\label{fig:figure_24}
\end{figure}

このような分割を楽曲全体に施行している菊地のような例はあまり多く見付けることはできない。ただし、21世紀に入ってからは、菊地・大谷の指摘するリズムの「揺らぎ」や「訛り」は、2012年にリリースされたロバート・グラスパーの''Black Radio''\cite{Glasper:02}を始め、数多く見られるようになり、これはポピュラー音楽に通底するアフリカ的リズム感の今日的な感覚の表われとして、菊地・大谷の定義の証左になると言える。

% ビジェイアイアー
%
% クラクラ
%
% これら総じて「揺れ」「訛り」をもった「ポリリズム的なリズム感覚」といっていいだろう。

% \subsection{考察}
%
% ーーーー執筆内容ーーーー\\
% 「ポリリズムやクロスリズムに関して人類はすでにこのようなことをやってきた。そして、それは事実、身体化可能なものだった」ということを説明したい\\
% クラシックの小節線ずれこみなども、基本的に周りこむ場合には拍が共有される場合が多く、アフリカンポリなどの訛りがあるポリの常態では小節線が共有される場合が共通して見られることを指摘する。そこに現代的なグルーブ感やリズム認知があると考えることが出来る。そこにはクラーゲスが言う「拍子(タクト)が同一のものを反復するとすれば、リズムでは類似が回帰する」ということに対するリズム回帰する力がポピュラー音楽に於いて強く働いていると仮説をたてる。\\
% ーーーー執筆内容ーーーー\\
%
% ここまでポリリズム/クロスリズムというリズムについてのアプローチを見てきた。
%
% またこの事は、五線譜による記譜法が17世紀以降に確立され、同時期に平均律による調性と和声に重きが置かれるようになったバロック時期には見られなくなったことを

% \section{現代の音楽制作環境と考察}
%
% 次に、先のポリリズム/クロスリズムの状況を現代の音楽制作環境から考察していく。考察にあたってはポピュラー音楽に於ける楽理、楽譜、制作環境としてのソフトウェアを取り上げる。
%
% \subsection{楽理的考察}
%
% ーーーー執筆内容ーーーー\\
% ハーモニックリズム、ドミナント進行、コードの共有、が一定のリズムフィギュア、もしくは小節という単位、にのってる点をバークリー理論や作曲法を引き合いに出して述べる。このあたりはザックスの本から拾えないか??\\
% ーーーー執筆内容ーーーー\\
%
% バロックでのリズムと和声の密接と4分の4拍子の支配的
%
% バークリー理論でのポピュラー音楽への展開
%
%
%
% \subsection{楽譜的考察}
%
% ーーーー執筆内容ーーーー\\
% で、上記の話を記述する楽譜自体が「拍と小節線の共有」という、所謂「行進」のような話とそういう音楽を記述する上で発展してきた背景が、システムとして全体を縛っている、という話を進める。ここは先の4/4が支配的になった話をして譜面のシステムとしての完成と自由度のバーターを論じる。\\
% ーーーー執筆内容ーーーー\\
%
% \subsection{ソフトウェア的考察}
%
% ーーーー執筆内容ーーーー\\
% その楽譜の考えかたで現代のソフトウェアは出来ているので、多様性が生じ難い土壌になっている。その上でそういう音楽がマスで流通するのだから、より4/4の支配力が増してるだろう事を指摘する。\\
% ーーーー執筆内容ーーーー\\

%改ページや図版を章に収めたい場合は以下のように改ページすると収まる。
\afterpage{\clearpage}
\newpage

\section{多層時間構造による音楽}

以上のポリリズム/クロスリズム関するリサーチをふまえ、作品制作にあたっては以下の点に留意して作曲を進めた。

\begin{enumerate}
  \item ポリリズムの定義である「整数で割りきれないビートを持つこと」
  \item シンプルなルールでそれを可能にすること
	\item グルーブ感のコントロールが出来ること
\end{enumerate}

これらの留意点は、筆者が2016年に発表した複数のリズム周期的に周りこむ楽曲である''{\bf New York 2997}''\cite{yamato:05}という作品を念頭に置いている。ポリリズム/クロスリズムはここまでの説明でも解るように「ある小節内でリズムが揺らぐ」状態が多く見られるが、この小節内に限定された動きを小節線を超えて行うことを試みた。筆者が序論で述べたように4拍子が多くの楽曲で支配的であるならば、その4拍子の使い方を変えることで、既存のポリリズムとも違ったリズム感が提示できるのではないかと考えたからである。

その為の試作として、まずAbleton Live上で動作するプラグインを''dc.PolyDelay20''を開発した。図\ref{fig:figure_25}にそのインターフェースのスクリーンショットを示す。

\begin{figure}[ht]
\centerline{
	\includegraphics[
 width=\textwidth]{./PDF/figure_25.pdf}
}
\caption{dc.PolyDelay20 のインターフェース}
\label{fig:figure_25}
\end{figure}

このプラグインは、MIDI情報の入力をトリガーにしている。その入力をディレイをつかって特定の拍の長さの中で最大で20箇所を指定して発音させることで拍の分割を行いかつ、特定のタイミングで発音を行うことができる。またその際に音程情報を可変できるので、フレーズを生成することが出来る。このプラグインをつかって試作を重ねた。この時にある拍数を4拍に分割するという作曲の試作として''{\bf ピアノ・エチュード}''\cite{yamato:06}と''{\bf パーカッション・アンサンブル}''\cite{yamato:07}を制作した。これらの曲は4拍から13拍までの長さの中で、各々を4拍に分割したフレーズの積層で作られている。それぞれの曲のフレーズについては図\ref{fig:figure_26}(こちらのURLから同じものをダウンロード出来る。 \url{https://goo.gl/g3EvW1})と図\ref{fig:figure_27}(こちらのURLから同じものをダウンロード出来る。\url{https://goo.gl/J23Pwi})にて示す。

\begin{figure}[ht]
\centerline{
	\includegraphics[
 width=\textwidth]{./PDF/figure_26.pdf}
}
\caption{ピアノエチュードのフレーズ譜。上から13拍4拍子、11拍4拍子、9拍4拍子、7拍4拍子、6拍4拍子、5拍4拍子、4拍4拍子の各フレーズになっている。}
\label{fig:figure_26}
\end{figure}

\begin{figure}[ht]
\centerline{
	\includegraphics[
 width=\textwidth]{./PDF/figure_27.pdf}
}
\caption{パーカッション・アンサンブルのフレーズ譜。上から4拍4拍子、5拍4拍子、6拍4拍子、7拍4拍子、13拍4拍子の各フレーズになっている。}
\label{fig:figure_27}
\end{figure}

この譜面は各フレーズを4分の4拍子として記述した場合の譜面であり、実際の発音タイミングとは異っている。それを実音価に直したものが図\ref{fig:figure_28}(こちらのURLから同じものをダウンロード出来る。\url{https://goo.gl/694iHz})と図\ref{fig:figure_29}(こちらのURLから同じものをダウンロード出来る。\url{https://goo.gl/i6J5TM})になる。

\begin{figure}[ht]
\centerline{
	\includegraphics[
 width=\textwidth]{./PDF/figure_28.pdf}
}
\caption{図\ref{fig:figure_26}のピアノエチュードを実音価に直したフレーズ譜。}
\label{fig:figure_28}
\end{figure}

\begin{figure}[ht]
\centerline{
	\includegraphics[
 width=\textwidth]{./PDF/figure_29.pdf}
}
\caption{図\ref{fig:figure_27}のパーカッション・アンサンブルを実音価に直したフレーズ譜。}
\label{fig:figure_29}
\end{figure}

このように4小節線を超えて4分割を行うことで、通常の楽曲では見られないような非常に細かい粒度の音価での揺らぎや訛りをシンプルなルールで作りだせることが解った。こういった取り組みの通じて考案したのが、次に説明する「多層時間構造による音楽」である。

%改ページや図版を章に収めたい場合は以下のように改ページすると収まる。
\afterpage{\clearpage}
\newpage

\subsection{「多層時間構造による音楽」の定義}

「多層時間構造による音楽」は、大きく次の二つの要素によって規程される。

\begin{enumerate}
  \item 異ったテンポのライン(メロディーやリズム、器楽パート)が同時に流れること
  \item 1 の異ったテンポのラインが「定期的にある小節の先頭の拍で揃う」こと
\end{enumerate}

端的に言ってしまえば「異ったBPMの$ n $拍4拍子の複層で作られた音楽」ということになるが、いわゆる無作為なマルチBPMやフリーテンポとは違い、ある周期で拍が揃うのが特徴である。

\begin{figure}[ht]
\centerline{
	\includegraphics[
 width=\textwidth]{./PDF/figure_01.pdf}
}
\caption{4拍4拍子の共通拍から5拍4拍子を作る過程を示した図。4拍4拍子の1拍を5つ並べることで5拍5拍子を作ることができる。その長さのまま4連符化することで5拍4拍子にすることができる。また4拍子の基礎グリッドとして1拍を4つに割れば5拍4拍子に於ける16分音符のグリッドを作ることができる。}
\label{fig:figure_01}
\end{figure}

図\ref{fig:figure_01}は、4拍4拍子と5拍4拍子の複層の場合の5拍4拍子の考え方を示している。まず4拍4拍子(つまり通常の4/4拍子)の1拍の長さを元に5拍分の長さを取る。それをその長さのまま4拍化(つまり5拍4連と同義になる)する。その時、小節線を共有することなくどちらも4拍子として扱うことでテンポの違う4拍4拍子が複層する。さらにそれぞれの4拍4拍子は元はそれぞれが4拍と5拍の長さであり、1拍の長さを共有してることから、4拍4拍子側の20拍目にあたる5小節目の4拍目の次の小節の頭拍(21拍目)と、5拍4拍子側の16拍目にあたる次の小節の頭拍(17拍目)で拍の打点が揃うことになる。これは単純な最小公倍数ではあるが、双方の$ n $拍4拍子はその間の小節の頭では拍は揃わず、最小で16分のずれこみを起しながら周回することになる。

このようなシンプルな方法を使って$ n $拍4拍子を複層しているのが「多層時間構造による音楽」の基本的な考え方である。

\subsection{「多層時間構造による音楽」における記譜}
「多層時間構造による音楽」における記譜については、先の試作の例を見て解るように通常の記譜では難読化してしまう。そこで、作曲者主観を明確に示し、かつ可読性を担保する必要があると考え「可変拍表記」「固定拍表記」という表記を用いている。以下、それぞれについて作品 ''Music for multilayered temporal structures'' に含まれる楽曲を元に解説する。以下の図\ref{fig:figure_02}は、同作品に含まれる ''Etude #1'' という楽曲のDrumパートにあたる。

\begin{figure*}[htb]
\centerline{
	\includegraphics[
 width=\textwidth]{./PDF/figure_02.pdf}
}
\caption{Etude #1 INTRO部のDrumパート}
\label{fig:figure_02}
\end{figure*}

図\ref{fig:figure_02}の譜例は上2段で4拍4拍子と5拍4拍子の複層の「可変拍表記」の譜面が示されている。可変拍表記の特徴は、この譜面の場合 $ n $拍4拍子の複層となるので、4拍4拍子と5拍4拍子とでは小節線が共有されずフレーズが周り込んでいく。また5拍4拍子側は所謂「5拍4連」と同等なのだが、通常の連符音価では「小節線が周りこみ、基礎となる拍の長さが違っている」という $ n $拍4拍子のコンセプトを示すことが出来ないので連符記号で「4:5」の様に示し、それぞれの4拍子の音価に表記の上で揃えている。

それに対して同譜例の3段目にある「Fixed Score」と指示されている部分が「固定拍表記」にあたる。これは上2 段の「可変拍表記」を小節線の揃った形(ここでは4拍4拍子) に纏めたもので、所謂一般的な譜面の書式にあたる。

また同作品に含まれる ''Etude #3'' では次のような図\ref{fig:figure_03}のような表記をする必要がある。

\begin{figure*}[htb]
\centerline{
	\includegraphics[
 width=\textwidth]{./PDF/figure_03.pdf}
}
\caption{Etude #3 [A][B]パート}
\label{fig:figure_03}
\end{figure*}

''Etude #3'' は、3拍4拍子、4拍4拍子、5拍4拍子の3層で作曲されている。これはその「可変拍表記」の例だが、3拍4拍子(譜例1段目)と5拍4拍子(譜例3段目)は先の ''Etude #1'' と同様にそれぞれ5小節、3小節で小節の終端が揃う。それに対して4拍4拍子(譜例2段目)は小節線の途中で次の段に折り返す必要性が出てくる。

この様に本来譜面の一般的な書式では読譜上、禁則として扱われる記譜が必要となることが「多層時間構造」を持つ音楽の一つの特徴であり。それが譜面という「演奏可能性を担保するシステム」や「楽曲制作に関する環境そのもの」に対しての問題意識を生み、同時にリズムから生れる「多様性を持った新しい音楽体験的特徴」にも繋がる。

\subsection{「多層時間構造による音楽」におけるリズム}

次に「多層時間構造による音楽」におけるリズムについて解説する。

まず、先に述べたように「多層時間構造による音楽」におけるリズムの基礎的な周期は「$ n $拍」の複層であり、つまり"Etude #1"、"Etude #2"のような4拍と5拍単位の複層なら20を最小公倍数とした周期と考えることができる。これは先のアフリカの3:4のリズムが8分の12拍子をそれぞれ 3+3+3+3 と 4+4+4 で示すことが出来るのと同じく、4:5のリズムなら一小節に換算すれば16分の20拍子として、それぞれを 4+4+4+4+4 と 5+5+5+5 の垂直ヘミオラの形を取ることになり、これは基本的なアフリカのポリリズム(本論文の定義ではクロスリズム)の形と原理的には同じ構造を持っている。"Etude #3"の3:4:5であっても最小公倍数が60とフレーズの周りこみが揃う周期は4:5に比べて大きくはなるが、そのままでは垂直ヘミオラであることには変りがない。この垂直ヘミオラからポリリズム的な揺らぎを生む為に、それぞれの$ n $拍を4拍子化した。

具体的な打点の計算方法は次の通りだ。「ポリリズム的な揺らぎを持つ$ n $拍4拍子の複層が小節の頭拍地点で揃う」拍$ N $は、それぞれの$ n $拍の最小公倍数$ LCM $に1を足した拍がそれにあたる。次に最小公倍数$ LCM $を$ n $拍で割りることで小節数$ bars $を計算できる。さらに小節数$ bars $に4を掛けることで各$ n $拍4拍子が頭拍で揃うまでの4分音符の数$ beats $が解る(8を掛ければ8分音符の数が解る)。以上のことは次の簡単な数式で示すことが出来る。\\

\begin{math}
beats = LCM \div n \times 4
\end{math}\\

求められた$ beats $で$ LCM $を割ることで各$ n $拍4拍子の一拍の長さ$ length $が計算できる。\\

\begin{math}
length = LCM \div beats
\end{math}\\

これに$ beats $の1から最大値$ m $までを掛けることで各拍の位置$ p $を$ LCM $から計算できる。\\

\begin{math}
(p_1,p_2\dots,p_m) = length \times (beats_1,beats_2\dots,beats_m)
\end{math}\\

その値を相対化し各$ n $拍4拍子を比較することが各$ n $拍4拍子の共通の拍を求めることができる。実際の比較には初期はグラフ用紙を持ちいた手計算を行い、後にopenFrameworks上で実装したプログラムを利用した。その比較から得られた共通の打点を作曲上のリズムのモチーフとして使っている。使用したプログラム''4by5''\cite{yamato:02}及び''3by4by5''\cite{yamato:03}は筆者のGitHub上のレポジトリにて公開されている。

また、$ n $拍4拍子の複層というコンセプトは先にも述べたように「異なったBPMの4拍子」の複層と言うこともできる。この場合の各$ BPM $は、基礎となるBPMを$ BaseBPM $としたときに次の式で求めることができる。\\

\begin{math}
BPM = BaseBPM \div ( n \div 4 )
\end{math}\\

例えば''Etude #1''の場合、115BPMで基礎となる4拍子の速さを指定している。上記の式に当て嵌めれば、4拍4拍子が115BPMであるのに対して、5拍4拍子は$ 115 \div ( 5 \div 4 ) = 92 $となり92BPMであり、整数比5:4のBPMの複層であることが解る。ただしこの115BPMの4拍4拍子に対して、3拍4拍子のBPMを計算すると$ 115 \div ( 3 \div 4 ) = 153.333\dots $となり、BPMの複層は整数比では捉えられなくなる。基礎となるBPMが120などであれば全て整数比を作ることが出来るが、基礎となるBPMによっては整数比ではなくなる。このことからもこの手法がシンプルなルールを持ちいて、プログラムなどを使う場合には単純なBPM換算では処理できないポリリズム性をもったリズムの打点を作りだせる方法であることが解る。

次に具体的なリズムの打点の例として ''Etude #1'' ''Etude #2'' の「4拍4拍子」「5拍4拍子」で使ったリズムのモチーフを図\ref{fig:figure_04}に示す。''Etude #1'' ''Etude #2'' ではリズムのモチーフは8分音符で比較された共通の拍を用いられている。緑で示されるのが共通の打点にあたるが、その拍だけを使った場合4拍4拍子側のグルーブが失われてしまうため、4拍4拍子の頭拍もリズムのモチーフに加え4拍4拍子と5拍4拍子の複層を表現している。またこの打点を「可変拍表記」で記譜したものを図\ref{fig:figure_05}に示す。

\begin{figure*}[htb]
\centerline{
	\includegraphics[
 width=\textwidth]{./PDF/figure_04.pdf}
}
\caption{''Etude #1'' ''Etude #2'' リズムモチーフ図。図の上が4拍4拍子の8分音符の打点を示している。下が5拍4拍子の8分音符の打点を示している。それぞれの緑のラインがそれぞれの共通の打点を示している。赤いラインはリズムのモチーフとして恣意的に追加された打点を示している。ラインの下に表記さている数字は、それぞれの打点の位置を数値化して示したもの}
\label{fig:figure_04}
\end{figure*}

\begin{figure*}[htb]
\centerline{
	\includegraphics[
 width=\textwidth]{./PDF/figure_05.pdf}
}
\caption{図\ref{fig:figure_04}を可変拍表記で示した例}
\label{fig:figure_05}
\end{figure*}

つぎに ''Etude #3'' での「3拍4拍子」「4拍4拍子」「5拍4拍子」の3層の楽曲で使ったリズムのモチーフ図を図\ref{fig:figure_06}にて示す。実際の楽曲では「4拍4拍子」は作曲の意図上、共通の打点は使っていない。

\begin{figure*}[htb]
\centerline{
	\includegraphics[
 width=\textwidth]{./PDF/figure_06.pdf}
}
\caption{''Etude #3'' リズムモチーフ図。上から3拍4拍子、4拍4拍子、5拍4拍子の4分音符での打点を示している。赤が3つの拍子に共通する打点、オレンジが3拍4拍子と4拍4拍子に共通する打点、緑が4拍4拍子と5拍4拍子に共通する打点、青が3拍4拍子と5拍4拍子に共通する打点を示している。図版の表示上見えないが、図\ref{fig:figure_04}と同じように各打点の下には数値化した打点の位置が示されている}
\label{fig:figure_06}
\end{figure*}

この共通の拍をリズムのモチーフに使うという手法は、各$ n $拍4拍子間の関係を作り、それによって演奏に際して構造的な理解をすることが可能になり、それぞれのリズムのパルスを各奏者自身によって生むことが出来る。図\ref{fig:figure_07}で、作品''Music for multilayered temporal structures''にて解説された「可変拍表記上でリズムの構造的理解」を行う為の「打点の取りかたについて」の説明を示す。

\begin{figure*}[htb]
\centerline{
	\includegraphics[
 width=\textwidth]{./PDF/figure_07.pdf}
}
\caption{''Etude #2'' 「打点の取りかたについて」}
\label{fig:figure_07}
\end{figure*}

この図で示されているように、共通の拍の関係性から自分の拍のテンポを生み出し、また「可変拍表記」を時間軸上で読譜することが可能になる。

%改ページや図版を章に収めたい場合は以下のように改ページすると収まる。
\afterpage{\clearpage}
\newpage

\section{作品 ''Music for multilayered temporal structures'' について}

この作品は先に述べた「多層時間構造による音楽」の手法を用いられて作曲された「エチュード」とその楽曲を演奏するにあたっての「指示書」の二つが主な内容となっている。ここでの楽曲は所謂「リードシート」の形式で記述されている。図\ref{fig:figure_09}に''The Realbook 5th Eddition''に含まれる''All the things you are''という楽曲のリードシートを示す。

\begin{figure}[H]
\centerline{
	\includegraphics[
 width=0.5\textwidth]{./PDF/figure_09.pdf}
}
\caption{''All the things you are''リードシート}
\label{fig:figure_09}
\end{figure}

このようなリードシートは「調性」「拍子」「メロディー」「コード進行」といった楽曲の構造を示したメタ記述としての楽譜であり、The Realbookなど広くジャズやポピュラー音楽の中で活用されている音楽の記述形式の一つである。そこで本作品では、リードシートをポピュラー音楽として作品を成立させる為の手法として採用した。またリードシートを採用し、ポピュラー音楽の基本的なフォーマットである楽器編成での再現可能性を問うことで、リードシートが持つ歴史的背景から今後の音楽の身体性を考えるアプローチに有効であると考えた。

\subsection{リードシートの位置付けと作品コンセプト}

「{\bf 1885年、ベルリナーが円盤型録音媒体を発明}」\cite{murai:01}した後「{\bf 1920年に始まったラジオ放送は短期間で全米に普及し、24年にはアメリカ全土で1200以上の放送局が存在するようになった}」\cite{murai:01}が、それ以前の音楽流通について村井は次のように述べている。「{\bf 「ホーム・スイートホーム」は(中略)1823年に発表されて、楽譜が何百万枚も売れる大ヒットになりました。(中略)ヒットというのは、楽譜の売れ行きを基準にしていた}」\cite{murai:01}。これはシート・ミュージックが「{\bf レコード以前の、最初に商品化された大衆音楽}」\cite{kikuchi-ootani:01}であることを示している。また『憂鬱と官能を教えた学校』に掲載された石塚 潤一による「シリンガーとバークリーの理論を巡って」という論考で1954年のバークリー音楽学校(後の音楽院)とバークリー・メソッド登場以降を次のように述べている。\\

{\bf つまり、バークリー・メソッドとは、クラシックの和声理論をもってブルーズを捉え直し、コードという新しい概念に合う形で徹底的に構造化されたものなのだ。(中略)クラシックの音楽理論もポピュラー音楽の実践に適った形で噛み砕かれていく。(中略)そしてこの新しいメソッドを後押ししたのが、他ならぬ録音記述の発展であったことも忘れてはならない。録音技術によって、細部まに至る記譜がなされていない即興的な音楽が、一回限りの演奏で霧散してしまうことから救われ、何度もプレイバックすることも、放送にのせることも、何万枚もレコードをプレスすることで世界的な市場を作ることも可能となった。これによって、記譜を身上としたクラシック音楽の絶対的優位が解体され、バークリー・メソッドのような隙間の多い理論体系に基く音楽が、世界を席巻する素地が作られていく。録音技術によって一般化された理論。この出自故に、これに続くポピュラー音楽の録音技術に代表されるテクノロジーの問題と切り話し得ないものとなる}\\

ここで石塚が「細部まに至る記譜がなされていない即興的な音楽」と呼んでいるのは当時のポピュラー音楽としてのジャズを指しているのは言うまでもない。また、そのジャズの今日的な発展にもバークリー・メソッドは一般的に使われる音楽理論として存在している。その時に、本作でターゲットにしたのが「リアル・ブック」と呼ばれるリード・シート集である。先にも述べたように、リード・シートは「調性」「拍子」「メロディー」「コード進行」といった楽曲の構造を示している。これは、先の石塚が指摘しているように「コードという新しい概念に合う形で徹底的に構造化された」ものだ。この構造化は、例えば先に挙げた図\ref{fig:figure_09}に挙げたリードシートで示された''All the things you are''という楽曲での、様々なミュージシャンによって演奏された実例を聴くことでも解る。ただし、この時に「この曲が''All the things you are''である」ことを保証するのは、このリードシートに示された様な構造の一致であって、正確な音韻、音律の再現ではない。この事は、実際の演奏にあたれば「どれひとつとして同じ演奏はなく、すべて同じ曲である」と認識出来ることからも解る。これを次のように言いかえることができる。\\

{\bf リードシートとは、身体性の違いによる表現の多様性と構造的一致による再現性の担保 = メタ作曲としての楽譜}\\

そこで、メタ作曲としての楽譜というリードシートの位置付けと、そのリードシートを使った所謂「スタンダード曲集」としてのリアルブックを標的とした。作品にリードシートを用いることでポピュラー音楽としての音楽と身体の関係を担保しつつ、次の音楽と身体の関係を考えることを可能とし、作品中にある「Beyond The Real Book」というコンセプトを立てた。

\subsection{作品構成}

この作品は下記の内容で構成されている。

\begin{enumerate}
  \item 作品コンセンプトの説明
  \item 「多層時間構造による音楽」の説明
	\item ''Etude #1'' 演奏指示書
	\item ''Etude #1'' リードシート
	\item ''Etude #2'' 演奏指示書
	\item ''Etude #2'' リードシート
	\item ''Etude #3'' 演奏指示書
	\item ''Etude #3'' リードシート
\end{enumerate}

これは、演奏者が各指示書を読んだ上で「多層時間構造による音楽」のアンサンブルを習得する、という目的に立ち書かれている。各エチュードは具体的な習得目的を定めた上で作曲されており、段階的に演奏者の習得を促すよう工夫されている。

\subsection{各エチュードの作曲上の意図と、各エチュードの関係について}

''Etude #1'' ''Etude #2'' 共に「4拍4拍子」「5拍4拍子」の2層になっている。''Etude #1''ではメロディが「4拍4拍子」、ベースが「5拍4拍子」を演奏する。ドラムに対しては両拍子の打点の特徴を明示的に示す意図でリズムパターンが組まれている。従ってメロディを弾く奏者はドラマーの示す「4拍4拍子」から自らのパルスを生みだすことが出来、ベースパートを弾く奏者も同じように「5拍4拍子」をドラマーから感じてパルスを生みだすことが出来る。ただし、ベースパートの「5拍4拍子」は意図的に頭拍を抜いている為、空白部分のパルスをより的確に感じる必要がある。このエチュードの目的は「基礎的な多層時間構造による音楽の習得」としてるため、このような工夫が行なわれている。またDドリアンでの1モード楽曲になっているため、ハーモニックリズムを感じ難い作りになっていることも先の目的に合せて意図されたものだ。図\ref{fig:figure_11}に ''Etude #1'' のリードシートを示す。

\begin{figure}[htb]
\centerline{
	\includegraphics[
 width=\textwidth]{./PDF/figure_11.pdf}
}
\caption{''Etude #1'' リードシート}
\label{fig:figure_11}
\end{figure}

''Etude #2'' は、''Etude #1'' に比較してコード進行が行われ、ハーモニックリズムを感じやすくまた楽曲の形式もAABAという良く見られる形式になっている。ただしBパートでは、メロディーとベースがお互いの拍子の乗り換えを行う。またドラムによるリズムの明示も薄くなっており、先に示した図\ref{fig:figure_07}にあるようにお互いの構造的理解によって各拍子のパルスを生みだす必要がある。この楽曲では「多層時間構造による音楽の応用とリズムの乗り換え」を習得するという目的が設定されており、その意図で作曲が行われている。図\ref{fig:figure_12}に ''Etude #2'' のリードシートを示す。

\begin{figure}[htb]
\centerline{
	\includegraphics[
 width=\textwidth]{./PDF/figure_12.pdf}
}
\caption{''Etude #2'' リードシート}
\label{fig:figure_12}
\end{figure}

''Etude #3'' は ''Etude #1'' ''Etude #2'' からさらに音楽的違いを出し、難易度を上げる為に、「3拍4拍子」「4拍4拍子」「5拍4拍子」の3層で作曲が行われていることが大きな特徴だ。またこの楽曲では ''Etude #1'' ''Etude #2'' で見られたような、ドラムによる各拍子の補助を行っていない。従って ''Etude #2'' での図\ref{fig:figure_07}で示したのと同様に、各奏者は自分のパルスを楽曲から構造的理解をし、演奏を行う必要がある。また、この楽曲ではコード進行も周りこみを行なっており「多層時間構造による音楽が持つ多調性の可能性」を示している。図\ref{fig:figure_13}に ''Etude #3'' のリードシートを示す。各リードシートは作品''Music for multilayered temporal structures''\cite{yamato:01}内でも確認することができる。作品は次のURLからダウンロードすることができる。\url{https://goo.gl/o1pTEk}

\begin{figure}[H]
\centerline{
	\includegraphics[
 width=0.5\textwidth]{./PDF/figure_13.pdf}
}
\caption{''Etude #3'' リードシート}
\label{fig:figure_13}
\end{figure}

上記の各エチュードの関係を纏めると図\ref{fig:figure_10}のように、1から3に向って段階的に難易度が上がる設計になっている。

\begin{figure}[H]
\centerline{
	\includegraphics[
 width=\textwidth]{./PDF/figure_10.pdf}
}
\caption{各エチュードの関係}
\label{fig:figure_10}
\end{figure}

各楽曲はApple Logic Pro Xで制作され、シミュレーション音源を制作した。それぞれの音源については次のURLで聴くことが出来る。

\begin{enumerate}
	\item ''Etude #1'' \url{https://goo.gl/Bv1odQ}
	\item ''Etude #2'' \url{https://goo.gl/49g16U}
	\item ''Etude #3'' \url{https://goo.gl/mzgDqu}
\end{enumerate}

\section{身体化の可能性}

ここまで述べた作品 ''Music for multilayered temporal structures'' の実演可能性の検証をするため、''Etude #1'' ''Etude #2'' を対象に実際のジャズミュージシャンによる実演をレコーディングし映像記録を行った。

演奏に参加したミュージシャンは、ドラマーに吉島智仁氏、ベーシストに川本悠自氏、ピアノ奏者に金子将昭氏の3名による所謂「ピアノトリオ」編成での実演とした。この実演にあたって、楽譜や楽器編成に「リードシート」「ピアノトリオ」といった既存のフォーマットを使うのは、この「多層時間構造による音楽」を単なるシステムや実験的なアプローチではない「ポピュラー音楽としての新しい音楽」を問うことに繋ると考えたからである。

実演にあたっては以下のスケジュールで行った。

\begin{enumerate}
  \item ミュージシャン決定後、シミュレーション音源の配布、資料の配布を行う
  \item 10/22(日曜)に第一回リハーサルを行う(ドラム、ベースのみ)
	\item 10/29(日曜)にピアノも加わっての第二回リハーサルを行う
	\item 第二回リハーサルでベーシストとピアノ奏者からリクエストがあり、シミュレーション音源のApple Logic Pro Xのデータも配布
	\item 11/10(金曜)にレコーディングと撮影を行う
\end{enumerate}

レコーディングと撮影は所謂「レコーディングスタジオ」で、奏者は各ブースに分れてヘッドフォンで互いの演奏を聴きながらの演奏となった。またレコーディングに際してはクリックの類は一切使っていない。この映像記録は筆者のyoutubeアカウント上で{\bf 映像記録 ''Music for multilayered temporal structures''}\cite{yamato:04} として公開されている。

\subsection{リハーサルについての考察}

先に示した通りリハーサルは2回行なった。回数と参加者については演奏者のスケジュールの都合となっている。まず第一回のリハーサルでは、ドラム奏者、ベース奏者とのリハーサルになった。事前に音源と資料配布はしていたが当日まで演奏者は特にリハーサルにあたっての練習をしている訳ではない状態であったので、まず作品(冊子)の音読を行い譜面の解説を行った。当時の資料には「多層時間構造による音楽」の解説部分と「可変拍表記」の解説部分は含まれていなかった為、筆者による解説が必要だった。ただしリハーサル自体については作品コンセプトに照しあわせ「実際はこのリハーサルには筆者は存在しない」という設定だという説明をした上で演奏者の自主性を重要視し、可能な限り観察に務めることにした。これは第二回リハーサルでも同じように行った。

リハーサルの初期段階では、ベース奏者、ドラム奏者共に「固定拍表記」的な考えかたで進めようとしたが、上手くいかなかった。これは「5拍4連が彼等が普段演奏するような音楽には出てこないこと」「5拍4連で考えたときに連符が4/4拍子から見たときに小節線を跨ぐこと」「5拍4連の頭拍が休符になっている」ことなどを理由として挙げていた。そのため、先の図\ref{fig:figure_07}にあるような楽曲の構造の説明と、リズムを取る様子を実演して見せることで理解を促すことができた。このリハーサルではベーシストは最後まで頭拍を抜くことは出来ず、練習として5拍4連のすべての拍を弾く練習から始めていた。ドラマーは複数の拍の理解を行う為にまず5拍の2分割を基準に習得を進めていた。リハーサル後半で ''Etude #2'' に移ったときにはスムーズに移行することが出来、奏者から ''Etude #1'' の有効性を感じる旨の発言を得ることが出来た。

次に第二回目のリハーサルでは、ピアノ奏者が加わり実際のレコーディングに向けたリハーサルとなった。このリハーサルでは、ピアノ奏者も始めは「固定拍表記」的な取りかたになっていたが、同じように5拍4連を演奏することが困難であり ''Etude #1'' よりも ''Etude #2'' に難易度を感じていた。また、前回と同じく筆者はなるべく観察者を務める立場であることを伝えた上でリハーサルを進行したが、演奏者が演奏の難易度を下げる為に楽曲の構造的な変更を行った場合にはそれを指摘し、許可しなかった。例を挙げると ''Etude #1'' がDドリアンのワンモードで作られている為、曲のリハーサル記号のBの位置が解り難い為 {\bf ''So What''}\cite{miles:01}を例に出し、半音上に変更したい旨申告されたが許可しなかった。実際に一度試してみたところ、それを行うことで楽曲全体の構造は解りやすくなるが、''Etude #1'' の目的が「多層時間構造の構造的理解」であり、ミニマルな形で行う旨伝えた。またこの日はソロを取るところで4分音符より細かい演奏が殆ど見られなかった。これは ''Etude #2'' でのピアノ奏者が5拍4連を固定拍表記的に理解しようとしてるとこから1拍を4拍4拍子の8分音符5つで換算してる為、5拍4拍子の1拍を割ることが出来なかったことに起因していると考えられる。

\subsection{レコーディングと記録映像についての考察}

二回のリハーサルを通じていくつか課題を見付けることができた。先にも示したとおりベース奏者、ピアノ奏者は各々が練習を行ってレコーディングに望んでいる。''Etude #1''は比較的スムーズに進行したが、テイクの後半になると演奏が完奏出来なくなっていった。これは通常の楽曲より各自がリズムをキープする負荷が高い為、といった類の発言が見てとれた。''Etude #2'' は、ピアノ奏者が「ベースの打点とドラムの打点がピアノの1、3拍を必ず示している」ということに自ら気付く様子が見てとれた。ただ、ピアノソロに関しては4分音符より細かい音符の使用頻度が少なく、指摘したところ演奏が困難であるという発言をしていた。ベース奏者は''Etude #2''のBパートでのリズムの乗り換えをBパートの直前のテーマの拍から切っ掛けにして乗り換えていた。特にソロに関してはその部分を上手く活用していることが見てとれる。

映像全体を通じて、それぞれが違う拍子を生みだして弾いていることがピアノ奏者、ベース奏者の身体の動きや頭の動きなどから見てとれることが出来る。また「レコーディング中(演奏中)はどういう音楽になっているのかわからない」が、実際のプレイバックでは「ポップに聞こえる」という発言や「譜面化された{\bf ''Bitches Brew''}\cite{miles:02}のようだが、譜面でこれを再現しようという試みは類例を知らない」という発言があった。

レコーディングに関してはドラマーのドラムパターンなど幾つかの工夫や演奏上の構成の工夫(''Etude #1'' でのソロからテーマに戻る為のinterlude的展開など)はあったが、楽譜全体の構造の変更を伴うものではなく、あくまでジャズミュージシャンが実演にあたっての現場での工夫という程度に留まりつつ楽曲の持つ構造を再現することが出来たと考える。これらリハーサル、レコーディングを通じて「多層時間構造による音楽」の演奏可能性は証明することが出来た。

\section{考察}

\subsection{実演したミュージシャンへのインタビューの考察}

実演したミュージシャンへのインタビューを、それぞれ個別に2017年12月15日に吉島氏に、同年12月18日に川本氏に、同年12月19日に金子氏に行った。インタビューは下記の項目についてミュージシャンの主観を調査することを目的とし半構造化インタビューとして項目を事前に決めて行っているが話の流れを重視し自由に発言してもらっている。インタビューの内容は\ref{appendix}にあるのでここでは抜粋して考察していく。項目については以下の通りとなっている。

\begin{enumerate}
  \item 今回の作品の動画や音源を聴いての所感
  \item 作品の冊子ついて、内容が理解出来たか?
	\item 冊子についての感想や、なにか自分に役に立つ知見のようなものはあったか?
  \item 今までなにか類例となるような音楽を経験したことがあるか?
	\item 普段やっていることと比較して差異はあるか?
	\item 実演の難しさは何処にあると考えるか?
  \item 今回の作品や手法の発展性についてどう考えるか?
\end{enumerate}

\subsubsection{「今回の作品の動画や音源を聴いての所感」について}

吉島、川本からは実際に演奏が可能なのか? という不安が上っていた。これは吉島の「ああいう譜面の前例がない」や、川本が「譜面に書いてある内容もあるし、これは大丈夫かなと思った」と言及してることからも解る。ただ両氏ともに「意外に音楽になっている」「思ったよりちゃんとした音楽になっていた」と譜面と音楽のギャップを感じていた様だ。金子も同様に実演出来たことに関しては評価をしている。譜面と実際の演奏については、「三人があわさってどういうサウンドになっているかとか、どういうグルーブになるかまではわからない」(吉島)と言う指摘があった。川本も「コンセプトは理解してたんけど、演奏不可能なんじゃないのか」という点を挙げており、作品からの難易度を感じていたことが伺える。

\subsubsection{「作品の冊子について、内容が理解できたか?」について}

吉島、川本が「理解出来た」とする一方で金子は「読み込みが足りていないのでまだ入っていない」としている。これは金子がインタビュー中に「ビジネス的にアプローチしてる」と語っているように通常の「仕事」としての立場で来ているところから依頼を完遂する為に必要なことのみに着目するという視点で冊子にあたっていたことが伺える。

\subsubsection{「冊子についての感想や、なにか自分に役にたつ知見のようなものはあったか?」について}

ドラマーである吉島は今回の試みを「例えば一つのレールがあるとして、もう一本違う方向で流れているはずなのになぜか同じレールに乗っているような感覚」と表現し、それを「すでに自分の中では応用出来るようになって、他の現場でもそれを出すことができる」と語っており、身体化レベルでの理解が進んでいることが伺える。一方で川本は「ミュージシャンの音楽上の情報から言えば、弱冠説明過剰なところはあると思う。もっとシンプルでもいいと思う」とする一方で「ただ、ドラムの譜面が2段に分れてたりとかああいう工夫はとても良い(中略)音楽のレイヤーの理解には凄く良かった。これとこれを足してこうなってるんだよ、っていうのが書いてあるのは凄く良かったと思います。現場に出てくる譜面としてはいままで見たことがないようなものだったけど、プロジェクトの説明書という意味では充分だと思う」とし、今回の作品実演にあたっての冊子の一定の効果を認めている。

\subsubsection{「今までなにか類例となるような音楽を経験したことがあるか?」について}

ポリリズム/クロスリズム一般についての類例が数多くあることや自らが試みている経験について聞きとることが出来ているが、この作品の特徴であるリズムの取りかたや記譜などの特徴への類似は「類似してるものが過去にあった可能性」に留まっている。これは全ての音楽を確認できないことと、記譜方法からくるアナライズの可能性についての言及も聞きとれた。

\subsubsection{「普段やってることと比較して差異はあるか?」について}

吉島の「普段やっていることのほうが難しい」という答えが興味深い。その理由に「それぞれが自分達のことしかやってないような場合に、自分が繋げようとしても継ぎとめるのは不可能」であってそういう現場に比べて、今回のほうが「演奏者が繋がろうとしているほうが楽なのか?」と尋ねたところ、「そこがやっぱり音楽的であるかってことでしかないと思う。そういう意味では今回の曲のほうが僕にとっては、気持ちは楽でした。いくら個人の能力が高くても。自分勝手に演奏する人が集まった現場のほうが辛い」としていた。これは、今回のドラマーの役割がパートの継ぎ込みという比重が大きかったことも影響しているのではないかと推測する。また川本は、この研究の位置付けを「ボキャブラリーの研究」との理解を示し「一点に的を絞った文法の研究という意味だと思っていて、ほんとはやらなきゃいけないことだと思っているんですが、普段なかなかそこまで的を絞ってその一つに時間掛けられない。それだけにコンセプトは解っても演奏出来るかわからなかった。それだからすごく価値のあること」だと自分の普段の活動との比較で研究の価値について言及していた。

\subsubsection{「実演の難しさは何処にあると考えるか?」について}

金子は「リズムが複合的になっているから」という今回の作品の基本的なコンセプトを挙げている。その難しさについては「人間が二つのことやると50\%ずつではやれず、30\%くらいになって効率が悪いということを聞いたことがあって。そういう大変さだと思います」とする一方で、実際には「人間が今回のようなものを処理するときにマルチタスクにやってない気がします。ドラマーの人がマルチで処理してないのと近いんじゃないか」と考察していた。吉島はソロについて「だれかがソロを取るときにだれかが基盤になってくれればいいと思うけれど、今回はドラムにその比重が大きかったがそのパルスを誰が出すかによって難しさが変る」と言及している。その例として「たとえばドラムがソロを取りだして、リズムを誰に委ねるか。多分僕はそれを結構な勢いで破壊してしまうから、そうすることで難易度はさらに上ると思う」と演奏の難易度が関係性によって変ると考えいてることが伺える。これは、楽曲がメタ作曲であり演奏者の解釈によって幅が生れる演奏の質的な差によるものではないかと考える。また川本は身体性について先の「ボキャブラリーの研究」という解釈から「新しい言葉を覚えても、その言葉を自由自在に使えるようになるまでには、一度フィジカルに落しこむ必要がある」点を指摘し、それには時間が掛ることを指摘している。ただし、その時間の量については「3拍4連を今回のような分解能で考えたことなんか一度もない。つまり、普通に聞いて覚えて弾いてを繰返してるから弾けてたわけで、普段から5拍4連の曲ばかり聞いて覚えてたら多分できるんだろう」と普段の経験の比重が大きいと考えていること示している。これはリハーサルの時に金子、川本共に「5拍4連」の難しさを普段の演奏で弾く機会がないことを訴えていたこととも通じる。

\subsubsection{「今回の作品や手法の発展性についてどう考えるか?」について}

吉島は譜面のもつ再現性から吉島自身が普段の活動で行っていることについての理解が進んだことを挙げている。これは現在進行形の吉島の活動に影響を直接与えていることになり、吉島自身が発展/継承してくれる可能性を示している。川本は対位法やバルトークの四重奏団の例を出し、その応用可能性をインプロビゼーションよりも作曲の手法として認めていることが伺える。

\subsection{他ジャンルのミュージシャンへのインタビューからの考察}

演奏に参加していない他のジャンルのミュージシャンへのインタビューも行った。インタビューはインド古典音楽の演奏家であるサントゥール奏者の安藤 真也氏と、タブラ奏者の原口 順氏の両氏に同時に行っている。インタビュー項目は先のインタビューと一部を覗いて基本的に同じ内容となっている。インタビューの内容は\ref{appendix}にあるのでここでは抜粋して考察していく。

% \begin{enumerate}
%   \item 今回の作品の動画や音源を聴いての所感
%   \item 作品の冊子ついて、内容が理解出来たか?
% 	\item 冊子についての感想や、なにか自分に役に立つ知見のようなものはあったか?
%   \item 今までなにか類例となるような音楽を経験したことがあるか?
% 	\item 普段やっていることと比較して差異はあるか?
% 	\item 実演の難しさは何処にあると考えるか?
%   \item 今回の作品や手法の発展性についてどう考えるか?
% \end{enumerate}

\subsubsection{「今回の作品の動画や音源を聴いての所感」について}

Etude #2のリズムの乗り換え部分でのリズムの変化やどちらの拍でも取れる、という作品の持つ特徴についてのポジティブな言及が見られた。また安藤は自分の体験と重ねて、ソロ奏者としての難易度に言及しそれを可能にする為には「みんなが変拍子がまず最初にスタンダードに5拍子の曲が普通に一杯あって、そこで5になれててという時代がきたら、違うリズムでバッと入れていけることが段々可能になるのではないか」と指摘している。これは安藤の主観では現在の音楽状況ではスタンダードなリズムが今回の楽曲や安藤の普段の体験とは違う、ということを示している。

\subsubsection{「作品の冊子ついて、内容が理解出来たか?」について}

音源と楽譜を両方あたることで理解が出来たとしているが、一方で安藤は「作者と話をしてるので、その先入観があるので、まったく知らないで、これをパッとこれを見たときに理解できるかはわからない」とし、冊子の初見での理解については保留している。これは筆者もある冊子の理解には一定の音楽リテラシーが必要なことを理解しており妥当な指摘だと考える。

\subsubsection{「冊子についての感想や、なにか自分に役に立つ知見のようなものはあったか?」について}

本研究の試みを通じて演奏者が同時に二つのリズムを演奏する可能性と現在の行っていることについては「サントゥールやタブラのような両手が使える楽器だから可能なのですが、ここでもドラムがやってるような同時に左右の手で別の拍を演奏するというのではなく、ふたりの奏者がそれぞれ違うリズムで演奏する例は色々な例があるのだけど、一人の奏者が両手で別の拍をやるというアプローチは過去にはなかったです。安藤さんはそれを大和さんと会ってから、前からモヤモヤと考えていたことが形になったとは夏の時に言っていて、それを最近実際にやっています」というような言及が見られた。これは先の吉島の理解と現場での活用に似ている。また、打楽器奏者であることが共通してることも応用可能性に関係してるのではないかと推測する。

\subsubsection{「今までなにか類例となるような音楽を経験したことがあるか?」について}

「本研究の類例にあたるわけではないが」とした上で、ザキール・フセインとミッキー・ハートが76年に演奏してる例を原口が挙げていた。これは「曲の解説を読むと102.5ビートの中で一つのメロディーを作る例」であり、とても大きな周期でリズムを取っている例だ。ただ安藤は「ザキール・フセインのやっていることは、アレンジとか楽曲を変えたりはしてるんですけど、考えかたはインド音楽のことしかやっていないです。シャクティとかでもいろいろ変えてるけれどインド古典しかやっていない」とし、音楽の基本的な構造はインド古典の構造で考えられると述べている。

\subsubsection{「実演の難しさは何処にあると考えるか?」について}

安藤はこういう試み自体は本論文でも示したリズムの打点の表のようなものを作ることで難しくない、とした上で「習得までの時間が掛ること」を挙げていた。また、その難しさがアンサンブルと個人で違うとし、それを「一人で自分がそれをやろうとすると。集団でやるようになるのはもう少し簡単で、そのハメ方を覚えていけば、もう自分の中にそれがポンと入れば。その変り自分が出してる持ち駒の中に5拍子の中にはこうなるよねとか、あと7拍子はこうだよねとか、3はこうだよね、6、12はこうだよね、ていうのが自分の中にあればそこにぼんぼんハメていけるんですけど、一人がこのドラムの人みたいにバラバラのことし始めるのは時間が掛ります」というように説明している。この事は今回のリハーサルでの吉島と、川本、金子の関係に準えて考えると興味深い。吉島はベース、ピアノを繋げる為の役割を担っているので一人で二つのリズムの打点を示す、というところに難易度を感じておりそれが出来るようになると楽曲全体を聴くことができるようになっている。川本、金子は安藤が言うような「ハメ方」、言いかえれば関係性を習得することで演奏を行っていた。川本がインタビューで語っているようにそれがボキャブラリーにない為に難易度があったわけだが、安藤が指摘するように自分の中にそういうボキャブラリーがあることで容易になる、というのは川本自体も5拍4連への習熟として語っているのと合致する。

またアンサンブルでのお互いの聞き方についての言及も興味深い。安藤、原口がポリリズム的なものを演奏するにあたって「いつものときよりポリリズム的なトリック的なことをやってるときは打点があってないとダメなので相当聞いてる。元のビートをやってる側は、ひっぱりこまれちゃうので、相手を聴きながらでも、もうメトロノームの様にやるんですけど、アプローチしていってる側のほうがどちらかというと聞いてる。聴きながらフレーズを収めていく」と語っていて、ポリリズム的な演奏での打点への意識の持ちかたの状態の違いが伺える。

\subsubsection{「今回の作品や手法の発展性についてどう考えるか?」について}

原口が「自分の演奏にもこれが取り入れられる要素がある」と比較的自分の演奏に具体的に言及する一方で、両氏ともに「一見聞いて分析がちゃんとこうじっくり向きあわないと解らない」が「音楽として聞いたらすごい美しいとかそういう音楽がもっと広がってほしい」と述べているが、これは今回の作品が美しいかはともかくとしても「一見普通に聴けてしまうが、構造的には複雑なものが裏側にいる。それをポピュラー音楽の文脈で将来の音楽の発展を考える」という目的にも合致する意見である。

\subsection{作曲、制作環境についての考察}

本研究は、先の川本の言葉を借りれば音楽の「ボキャブラリーの研究」として作曲手法の中でリズムにその範囲を絞っている。そこから作品を通じて広く作曲を見ていくと、\ref{Western-Modern-AmericanNewMusic}で示したように、現代において、14、15世紀のクラシックの記譜法が固まる以前の作曲の自由度がもし残っていれば、このような音楽がありえたかもしれない。また、それは表記法に寄って見えなくなる/見える、ということから既存の音楽の中にも同じような類例があるかもしれないという可能性は、インタビューからも言えると思う。そこから作曲を考えたときに、現在の作曲手法が西洋音楽、ポピュラー音楽を問わず現在一般的になっている記譜の構造的な考えかたによって成り立っているのであれば、アフリカのリズムを西洋音楽が掴まえ切れなかったのと同じように、まだリズムについては作曲の手法として可能性が残っていると言えるだろう。

また、作曲という行為を記譜するという行為から考えたときに、リードシート以前の楽譜を「コンピューターのプログラムとコンパイラ」の関係にアナロジーするなら、リードシートは「構造化されたデータとAI」の関係のように捉えることも可能だ。この事は「譜面に対してその音価や音律に従って再現または表現する」という西洋音楽における身体とは別に、ポピュラー音楽において「構造を再現する身体」という身体性の違いとして考えることもできる。

もちろん様々な音楽家が現在もあらゆるアプローチをリズムだけでなく行っていることは自明であるので、このアプローチだけが新規性があるわけでも、すでにあるアプローチが風化してるということを指しているわけではない。ただ本論文に於けるポピュラー音楽の作曲、という行為を通じて来るべき未来の音楽の可能性を考えると同時に、現状についての批評的な試みになっていると考える
。

本研究が「あるボキャブラリーの研究」であるならば、まだ本研究の先には別のボキャブラリーへ、またはこのボキャブラリーから関連させて考えることが出来るボキャブリーへと発展、継承させて行くことが可能だ。それは、ポピュラー音楽史の中で、シュリンガーハウスからバークリー・メソッド、リディアン・クロマッチク・アプローチなどの音楽理論だけでなく、ジャンルとしてのビバップ、モード、フリージャズ、ノイズなどを通じて現在も継承発展しているポピュラー音楽の様々なスタイルがその証左になる。

今回「多層時間構造による音楽」として作られた作品に置いて、その制作過程で制作環境の未整備を見てとれた。例えば、記譜は筆者が調べた限りLilypondでのみ可変拍表記の譜面を「構造的に」記述することが可能だった。この事も記譜というものが現在あたりまえとされている音楽の構造を前提としていることの表われの一つだろう。また、音楽制作ソフトウェアに関しても、マルチBPMや、多層された拍子でのトラック設定が出来るようなソフトウェアを見つけることができず、デモ音源の制作にあたってはLogic Pro XのMIDIのストレッチ機能を使って半ば「つじつまを合わす」ような制作の方法でしか制作できなかった。もちろんMIDI分解能を駆使して計算の上、MIDIデータを配置していけば可能なのかもしれない。ただそれは、可変拍表記の記譜で示したような「作曲者主観の制作手法」とは到底言うことが出来ないだろう。本研究がその意味でも既存の制作環境についての批評的試みになったと考える。

\section{結論}

\ref{research_target}でも述べたように、この研究は「ポピュラー音楽」の今後の発展をその研究の目的とし、リズムに着目した研究である。今回考案した「多層時間構造による音楽」の手法は、現状の結果からは次のように言うことが出来る。まず「超絶技巧を持たずとも構造的理解による演奏が可能であること」、次に「このような手法でアンサンブル上難曲でありながらもポピュラー音楽のコンテキストの延長線上の音楽を生みだせること」、最後に「制作や理解に記譜や制作環境などの整備が必要であること」である。今回の演奏者はプロとしてのキャリアを持っているが、このような音楽を専門的に弾く活動を日常的にしているわけではない。が、今回はメタ構造としてのリードシートとその指示書、また構造理解を促すファシリテーションを通じて自らの理解で演奏を可能にした。また著者は本稿で述べた「多層時間構造による音楽」という手法を用いて ''Etude #2'' のような通常のフォームの楽曲や、''Etude #1'' のようなジャズのコンテキスト上の楽曲を作曲することに成功した。

楽曲を作成するにあたって筆者はすでにバークリー・メソッドを学んだ経験を持っており、それ故に先に述べた作曲に成功したのだが、そのバークリー・メソッドがポピュラー音楽を発展させた理由に「バークリー音楽院がその理論を広めるにあたって制約を持たせなかった」事がある。つまりバークリー音楽院で学んだ音楽家が自らが教える側になることでその理論が世界中に広まったのだ。それと同じようにこのような手法の研究は「オープン・アクセス」され「継承または発展する」事に大きな意味があると考える。その意味でも、本研究にあるインタビューから解るように限られた事例ではあるが本研究を切っ掛けに、他のミュージシャンが自らの音楽表現を更新しようとする試みに繋げることができた。これは本研究の目的としている「ポピュラー音楽から未来の音楽を考える」という視点に立ったときに研究の持つ新規性よりも大きな成果だと筆者は考える。\\

最後に本研究からの今後の展望を述べる。\\

まず、このような構造を持った楽曲からリズムの可能性を探っていくと為に、まず制作環境の整備を通じてここから得られるリズム認知や楽曲の構造的理解を発展させたい。そこには和声的な展開、またアルゴリズム化やAIなどを持ちいたコンピューターとの共創や、そのような環境を「演奏者や共作者として迎える」などが加わる可能性が考えられる。また、引続き今回の演奏者などとの演奏活動を行うことや、他ジャンルの演奏者へのインタビューやコラボレーションを通じて、リズム認知についての考察を継続したい。

本研究で立案した「多層時間構造」というコンセプトの展開については、次のように考える。まず、本研究で立案した方法から創りだすことが可能であろう新しい和声への展開だ。本研究で示されたシンプルな手法で複層されたリズムと共に和声の展開を考えることで、従来の和声の拡張とは別の手法として新しい和声の展開の方法が作れりだせる可能性がある。その可能性を示す一つの試みとして''Etude #3''では、同じコード進行のルールを二つの拍子に分けて作曲を行っている。このような方法を取ればシンプルな響きを持つトライアドが周りこみ、必然的に時間進行に併せて和声の響きが複雑化/多調化/無調化していく可能性があるからだ。インプロビゼーションなどではリードシートに書かれているコード進行とは別の和声を「想定」し奏者自らが拡張/解釈して演奏を行うことがあるが、そうした和声の拡張のコンセプトをリズムとの関係性で行うことで「多層時間構造」が持つ独自の和声の設計が作曲の段階からも可能となるだろうし、インプロビゼーションの方法としても用いれば従来の方法とは別のアプローチとして有効になる可能性がある。また、身体化の可能性としてダンスなどの身体表現との関係を取り上げることで、演奏同様に、単に鑑賞する側の体験ではなく、ダンサー自身がいかにポリリズム/クロスリズムを「身体化」していくか、そのプロセスを考察することが必要である。そのような試みを通じて「ダンスと、そのBGM」といった主従関係ではなく、共に「音楽の視聴体験を作る奏者」としての関係を作りだすことや、映像作品などの他のタイムライン芸術との関係に於いても同様に「身体感覚に於ける新しい視聴体験」を生みだす可能性がある。リズムから得られる音楽芸術の発展の余白はまだ充分に残っている。

\section{付録 \label{appendix}}
\subsection{参加ミュージシャンへのインタビュー1 (吉島 智仁氏、2017年12月15日)}

\begin{description}
 \item[今回の作品の動画や音源を聴いての所感]\mbox{}\\

 ああいう譜面の音楽の前例がない。作るのが大変なので、出来上がるまえどうなるのかわからなかった。動画になってみると意外と音楽になっている。そうすると動画を見るとあそこからまたアイデアを貰える感じだと思った。\\

 --- {\bf レコーディングのときに印象的だったのは「プレイしてるときにはどうなっているのかわからない」と言ってましたよね。}\\

 一回目のプレイバックから二回目のプレイバックまでの(プレイバックを聞いてるので)理解は速度が早い。ブースで「こういう風にしょう」という話合いが出来た。割りと出来上がってみて形が一回決まると、なにかコントロール出来るというか、プラモデルの出来た写真というか「これからこれを作るんだぞ」というのが、こうわかる感じがあります。譜面が説明書、というかレコーディングにおいては自分達が演奏したプレイバックが説明書みたいになった。\\

 --- {\bf それは冊子の内容がわからないとかそういうレベルだったのか、それとも冊子は冊子で解るんだけどやってみないと解らないという範疇なのか、そのあたりはどうですか?}\\

単順に譜面は読める。だけど三人があわさってどういうサウンドになっているかとか、どういうグルーブになるかまではわからない。今グルーブが一番難しいと思っていて、譜面にグルーブが書けないじゃないですか、揺らぎとか。そこにやっぱり目が行く。\\

--- {\bf あとドラマーって世の中的には和声楽器やメロディー楽器だと思われていないところがあるけれども、吉島さんがSNS上で、時折作曲者に対してグルーブがどうなってるのかという点、でもまわりの音に対してどう反応するかみたいなところを凄く気にしている。「なにが起っているの?」ということやアドリブだけじゃなくてリズムに対して関心を持って欲しいという気持ちもあれば、吉島さんが和声やアドリブのラインに関心や興味を持っている発言をしてる。そういう興味の持ちかたの構えかたというのがそれはドラマー全般なのか、吉島さんという人が独特なのかどっちなんでしょうか?}

単順に2パターンあると思っていて、元々ピアノとかギターをやっていた人は今メロディとかハーモニーがどうなっている気になるのか、あるいは耳が勝手に聴きにいくのかと思う。打楽器から始めた人で、しかも皮物とかだと、そちらには行かずにもっと別なところで聞いてる様な気はします。

\item [作品の冊子ついて、内容が理解出来たか?]\mbox{}\\

わからないといことは全然ない。普通に読めました。\\

--- {\bf 通常の譜面の書きかたでない部分もあったと思うが、それも関係なく全然読めるという感じですか?}\\

それぞれバラバラに読めば良いというだけで(読めた)、要は、それが同時に流れた時というのが難しいというだけで、一個一個書いてあることは全然解る。\\

\item [冊子についての感想や、なにか自分に役に立つ知見のようなものはあったか?]\mbox{}\\

例えば一つのレールがあるとして、もう一本違う方向で流れているはずなのになぜか同じレールに乗っているような感覚っていうのは、やっぱりああいう音楽をやらないと出てこないんじゃないかと思うし、それは演奏してるうちに聞こえるようになってしまった。だからすでに自分の中では応用出来るようになって、他の現場でもそれを出すことができるんですけど、それが最近の一連の事件の原因。\\

--- {\bf 現場で使ってるというのは凄いですね。}\\

使い始めたら、誰も解らないですね、やっぱり。でも人によっては相手は相手で挑戦するから果敢に「それを出したままにしてくれ」という人もいる。(拍を)数えることを許さない状態になってるというか、そもそも2拍目以降は数えられないですよね。こっちとしては慣れてしまっているから助走がなくても演奏できる。さきまで「イチ、ニ、サン、シ」って数えてたその四拍をそこからはみだした状態で4に割るという状態に慣れてしまったので、自分の中では変えることが出来る。「オープンソロをやってください」と言われたときに、僕としては「(相手に)後悔させるつもり」でそれをやる。\\

--- {\bf オープンソロとかになってしまったら拍頭がわからなくなっちゃいますよね。}\\

僕もそっちにいっちゃったから戻れないんですよね。知ったものを捨てることはできないでしょ? 慣れたら現場でもやってみたいと思う。\\

\item [今までなにか類例となるような音楽を経験したことがあるか?]\mbox{}\\

これに関しての類例は難しい。レコーディングの時に指摘したようにビッチェズ・ブリューを譜面に書いて、あの独特の雰囲気を作ろうとしたらこういう譜面が必要になるのかな、みたいな。だけど今回のような逆のパターンはまだ見たことがない。譜面に書いてあって、それを演奏した結果、ビッチェズ・ブリューのようになったという経験はまだなかった。ただああいう譜面を今回やってみたことで、例えば菊地さんがやってたこととか、ティポグラフィカとかやってたことがもしかしたらこれに近かったのかな、というのはある。\\

--- {\bf あの譜面の書き方を知ってるか、知らないかで聴く音楽の被分析性が変る部分があると思うけれど、ああいう譜面の書きかたすればもしかしたら「この音楽は分析出来るのでは?」というのはあると思うが、ポリリズム/クロスリズムの場合小節線の共有が前提にあって、周り込みの場合はビートの共有が前提にあって、どちらかしかないような気がしてて、今回はどちらもやっているが。}\\

僕はそこは意外とどちらもなくて、メロディーを一番に読みとって演奏するからどこで息継ぎするかしか考えてない。それに対して例えば5拍の音楽あるとして、音楽自体のリズムは5拍子になっていても息継ぐところが「イチ、ニ、サン、シ、ゴ、ロク」最後だとしたらそれを一つと捉えている。だからあんまり実際のところは拍子は数えていないかもしれない。全部1拍子。\\

--- {\bf それは(積んでいくという意味で  )インド音楽の人みたいですよね。}

だから、フランク・ザッパのブラック・ページとか演奏したことがあるけれど、なにが難しいのかわからない。あれはぜんぶ読めばいい。ハイハットは4分で踏んでるし。10何連符が難しい、とかは(技巧的に)解るんだけど。だから読めばいいだけなので難しいと思ったことがない。\\

--- {\bf ブラック・ページみたいな音楽は現代音楽的な超絶技巧曲で、そういった難しさに比較して、今回演奏していただいた曲はリードシートだけ見ると誰でも演奏できる。でも実際は演奏に際して今回みなさんが難しい難しい、と言っていたけれで、その難しさの違いってなんだと思いますか?}\\

それは今回の曲は、一つ一つのラインが難しいんじゃなくて、アンサンブルが難しい。例えば三人でやらなきゃいけないってなったら、全然違うものが聞こえてくるから、その中で自分が出さきゃいけないパルスを守んなきゃいけない。その上でブラック・ページと比べるなら、ブラック・ページは4分が決っている。そのうえに演奏者が乗って演奏するから、ブラック・ページは全員がああいう奇数連符を全員でユニゾンするのが難しい。だから一人でやったら難しくない。\\

--- {\bf 歌う気になれば自分一人でブラック・ページは歌えてしまう。}\\

書いてあるのを読めばいいだけだから。でも、今回の曲は書いてあるのを読んでも一つじゃ意味がない。だからアンサンブルの練習の為のエチュードであって個人の練習の為のものではないです。ところが、アンサンブルとして成立させる為にドラマーがなにをするかと言ったら、他のパートを継ぎ止めないといけない。だから真ん中に入って、どうやってああいうズレこんだ拍を自分の身体の中で一つにするかという、ひとりアンサンブルのようなのは始めて。\\

--- {\bf 継ぎとめるという話が出てきたが、外から見ていて三人の中で一番音楽がどうなってるかを把握していたのが吉島さんだったと思った。他の2人は「これが正しい」というのが微妙に解らないようなことを言っていたが、吉島さんだけが「今正しいかった」って言えるという状況が起きてて、それはパートの役割の違いがあると思う。}

\item [普段やっていることと比較して差異はあるか?]\mbox{}\\

逆に普通に叩くことを要求される現場のほうが難しい。それぞれが自分達のことしかやってないような場合に、自分が繋げようとしても継ぎとめるのは不可能。\\

--- {\bf まだみんなが繋がろうとしてる今回のほうが楽?}\\

そう。そこがやっぱり音楽的であるかってことでしかないと思う。そういう意味では今回の曲のほうが僕にとっては、気持ちは楽でした。いくら個人の能力が高くても。自分勝手に演奏する人が集まった現場のほうが辛い。\\

--- {\bf 今回、レコーディングを見ていて演奏が走らない(テンポがだんだんと上っていかない)ように思えたが、もしかしたら今回音楽ってそういうことを許容しないのではないかと思うがどうか? なぜなら皆が構造的なものを守ろうとしてる。もちろん上のレベルがあるので、今回だけでそう言い切れるとは言えないとはおもうというのが前提ではあるが。}\\

曲に対して慣れていけば、やっぱりチャレンジしたくなる。そのチャレンジのイメージで、走ったり、モタったりということが起きると思う。\\

\item [実演の難しさは何処にあると考えるか?]\mbox{}\\

あれを捉えた上で「ソロ」をやらなきゃいけない、というのがなによりも難しいと思う。だれかが「ソロ」を取るときにだれかが基盤になってくれればいいと思うけれど、今回はドラムにその比重が大きかったがそのパルスを誰が出すかによって難しさが変ると思う。たとえばドラムがソロを取りだして、リズムを誰に委ねるか。多分僕はそれを結構な勢いで破壊してしまうから、そうすることで難易度はさらに上ると思う。\\

\item [今回の作品や手法の発展性についてどう考えるか?]\mbox{}\\

難しいなぁ、もしかしたらもうあったのかもしれない。今回は譜面からそれを作る、特殊事例だけど、意外と聞いたらこれはこういう風に聞こえる、というのはあったから、目に見えて変るということはもしかしたらないのかもしれないけど、だけどそれを譜面でコントロールする、というのは恐しい。例えばいままでエレクトロニカで乱数的に入っていたような「適当に入ってるけど気持ちいいな」みたいなものが、もうコントロールされる。再現性が出てくる意味ではもの凄く進歩はすると思う。もう一度これが聴ける、というそういう強み。しかもそれをコンピューターに任せるのではなくて人間が演奏する。これは恐しい。

--- {\bf 今の話のポイントは譜面による再現性の話?}\\

僕みたいな変り者が喜ぶって感じ。適当に演奏していたときに「今日は調子わるかったなぁ」言う日あるじゃないですか? あんまり合わなかったね、みたいな。それが無くなる、というのは僕すごいストレスがなくなる感じがする。これをやれば自分が出したい音が出せるでしょ、っていうのが目に見えてわかるっていうのは、マイノリティーかもしれないけど、僕はすごく助かる。もう喧嘩しないでいい。\\

--- {\bf 吉島さんが現場で今回のリズムを発展というか継承して現場で使っている、というのは凄いことだと思うんですが、今回のこのプロジェクト全体でなにかここまでで出ていない話とかありますか?}\\

いままで自分がやってきた演奏って、僕の中でも結構ハッキリとしてないというか、自分の双方はいったいどういうところから来て、どこに着地するんだろうみたいな答の一つがこれだった。でも誰もそれを体系化できないし、自分の頭の中でもそれはシステマチックにならないことだったから。でも面白いのはいままでやってたことなのに、譜面で渡されてそれをやろうとすると出来ないんです。ピタっと合うようになったのがレコーディング前で、「これいつも自分のやってる演奏と対して変らないじゃん」って思って。\\

--- {\bf 今回の作品は新規性を強く主張するというよりも、ポピュラー音楽として同時多発的に発生する一つのやりかたを示しただけに過ぎない、と思っているのだが?}\\

チャーリー・パーカーが演奏していたことをバークリーがやったことと同じこと今起っていて、多分パーカーは自分の演奏が分析されなければジャズがあんなに発展しなかったと思うんじゃないかなと思って。\\

--- {\bf 歴史的にジャズが被分析性の高い音楽でそこにバークリーメソッドが嵌ったということで広く広まったということは強くありますよね。}\\

そうですよね。その分析が正しい正しくないは別にして、あれを足掛かりにしていろんな考えかたが出てきたじゃないですか? 場合によってはあれを否定するように、マイルスがモードを演り始めた。でもそうなったらもうカオスで、もう出てくるものがないからフリーに行ったという背景があるけど、でも実際出てきましたよね、こうやって。再現性がある、というのはもの凄く大きい。だけど誰でもやれるかという話は別だけど、やりたいかどうかも含めて。
\end{description}

\subsection{参加ミュージシャンへのインタビュー2 (川本 悠自、2017年12月18日)}

\begin{description}
\item[今回の作品の動画や音源を聴いての所感]\mbox{}\\

思ってたよりはちゃんとした音楽になっていたなぁ、という印象。それはプレヤビリティ的な問題もあるし、譜面に書いてある内容もあるし、「これ大丈夫かな」と思ったんだけど、でも思った以上に聞いてて心地いいものになったのでそれは良かったなぁ、と思うのと、同時に不思議というか興味深かったです。

--- {\bf 興味深いというのはどういったあたりでしたか?}\\

ああいうリズム・フィギュアがそもそもああいう楽理上のことを主眼に置いて書かれたものって、音楽的なことより優先してることがまずあるじゃないですか? 研究として目的があってやってることだからそれは良いのだけれど、それが意外に演奏してレコーディングしてみたら良いバランスになっていて、それは単に耳が慣れただけなのか、作品がそもそも良かったのか、演者のチカラなのか、その辺は解らないけども、でも出来上がったものに関してはとても満足しています。\\

--- {\bf 思ったより音楽的に成立してた、という言葉の裏には「成立しないんじゃないのか?」って思ってたということがあるんじゃないかと思いますが、成立しない状態っていうのはどういうものを考えてましたか?}\\

思ってました。大丈夫かなって。コンセプトは理解してたんけど、演奏不可能なんじゃないのか、というのはありました。それは一番大きかったです。その次は、研究上の制約が多いので、そこに書かれている音の羅列がメロディーに聞こえないんじゃないか、メロディーに聞こえないと音楽にならないから。でも聞こえてる。特にEtude #1のほうが「これは大丈夫かな」って思ってたんだけど、それは演者の力が大きいんじゃないのかな、と思っているんですけど。\\

--- {\bf 演者が併せてるだけなんじゃないの? という批判も貰ったりするのだがそのあたりはどう思うか?}\\

こういう研究のジレンマというか「演者が併せてるだけなんじゃないの?」っていう批判が出るって、音楽的な観点から言って「併せてなにが悪い」じゃないですか? ただ研究テーマとしてこういう試みが可能か不可能かというフィジカルな問題があったとしても、それは演者としての最終責任としてなにか形にしなきゃいけないっていうのが僕等沁ついてるから「出来ませんでした」ないし、自分のプライドも勿論あるし、研究に貢献いしたいという思いももちろんあるわけで、だからそこはその立場の違いで音楽上の現場の感覚っていうのは「併せてなにが悪い」。その辺が常にこういうことをやるには矛盾した立場として出ていくのかな、ってのは思います。それをどう作品として捻じ伏せていくのかってことなんじゃないのかな、と思います。\\

\item [作品の冊子ついて、内容が理解出来たか?]\mbox{}\\

もちろん理解はできました。十二分に説明はしてあるので。\\

\item [冊子についての感想や、なにか自分に役に立つ知見のようなものはあったか?]\mbox{}\\

この質問に真っ向から答えることにはならないけども、ミュージシャンの音楽上の情報から言えば、弱冠説明過剰なところはあると思う。もっとシンプルでもいいと思う。ただ、ドラムの譜面が2段に分れてたりとかああいう工夫はとても良いと思いました。家で練習するのに打ち込み音源を自分で作ってたのだけど、ああいうことは音楽のレイヤーの理解には凄く良かった。これとこれを足してこうなってるんだよ、っていうのが書いてあるのは凄く良かったと思います。現場に出てくる譜面としてはいままで見たことがないようなものだったけど、プロジェクトの説明書という意味では充分だと思う。あと認知的不協和の話は面白かった。大和さんらしいなと。\\

\item [今までなにか類例となるような音楽を経験したことがあるか?]\mbox{}\\

類例は山ほどあると思うけど、ジャズ全般そうだと思うし。\\

--- {\bf 一般的にはポリが起っている時には、小節線が共有されていて、フレーズの周りこみはビートが共有されている、という状態になってるが、今回の作品は小節線もビートも共有されてない。共有されるところは最小公倍数の拍で塊が大きい、ということを今回の作品の特徴としたときに同じような類例があるか? としたらどうか。}

それは難しい。クラシックの作品とかにならきっとあると思うけどあまりこれというのは言えなくて、でもジャズだとたぶんインプロの途中でそういうことは起りえるし、たぶんエヴァンスなんかもやっていたと思うし、マイルスバンドのトニー・ウイリアムスなんかが演っていると思う。\\

--- {\bf 今回可変拍表記みたいな譜面の書きかたをしてるけれども通常ああいう形でアナライズしない。アナライズ手法が西洋音楽やバークリーの書法みたいな形で行われているから解らなかったけど、可変拍表記でトニー・ウイリアムズなどをアナライズすると似たようなことを演っていることが解るんじゃないか、というのが想像できる。}

それはあるでしょうね。大和さんの言うように、バークリーまでに限定しないでも五線や小節線を使うこと自体が西洋のシステム、西ヨーロッパの民族音楽のシステムで、たとえばインド音楽などはそうじゃない書記をするわけで、そういう違うボキャブラリーで眺めたときに、違うアナライズの仕方やそういう構造の音楽に最も適した分析方法があるっていうのは全然不思議ではない。\\

--- {\bf アフリカンポリを西洋音楽の楽譜で採譜したときに西洋の音楽はアフリカンポリを取り逃してるのは塚田健一の「アフリカ音楽の正体」で書かれていて、どういうビットマップというか構造で見るかという目線がないことで取り逃がしている。}\\

それで言えば、ジャズをやってる身からすると「スウィング」のリズムっていうのは五線譜に記譜不可能なので、そもそもあんまり譜面を信用していない、というスタンス。あとジャズにはビートのタイプが5種類ある、というのがあって、パーカーもそうだし、素晴しい奏者っていうのは一つのパルスに対して自分がどの立ち位置で演奏してるのかというのをしっかり意識してコントロールしている。レイドバックもレイドバックって言ってるけど、レイドバックには二つあってそれをコントロールしてる。なぜジャズがそうなっていったかというと、アフリカの人達の民族的な微妙な楽譜上のズレのコンセプトがそのままジャズに受け継がれていて、それは譜面ではなく恐るべきことに口伝で伝わってるもの、ていうことだという事実じゃないかと思う。それはジャズの民族性みたいなものを考えたときに黒人特有のものだと思う。日本人はそれほど連続するリズムに興味がないから。\\

--- {\bf ロバート・グラスパーなどのリズム感も通じるものがあるのではないか?}

それも僕は過去の文脈の中から出てきた発想だと思っていて、ジャズ・ミュージシャンが親和性が高いのは根っこが一緒だからで、そういう文脈の中から出てきたのではないかと思う。ガンサー・シュラーの「初期のジャズ」という本を読んだのだけど、ガンサー・シュラーはアフリカの民族的なリズムの細かいパルスの微妙なズレをたまたまズレてるのではなくてちゃんと、チャーリー・パーカーやディジー・ガレスピーなどのジャズ・ジャイアントもそれを認識しながらやってるというようなことを書いている。\\

\item [普段やっていることと比較して差異はあるか?]\mbox{}\\

今回は「ボキャブラリーの研究」だと言えるのではないかと思う。なにか作曲をするにしてもプレイをするにしてもボキャブラリーはとても大事だけれども、それ以上に「なにをしゃべりたいんですか?」みたいなのが、音楽を作る場合にはその全てを綜合してやるわけだけれども、その中で一点に的を絞った文法の研究という意味だと思っていて、ほんとはやらなきゃいけないことだと思っているんですが、普段なかなかそこまで的を絞ってその一つに時間掛けられない。それだけにコンセプトは解っても演奏出来るかわからなかった。それだからすごく価値のあることだと思いました。\\

\item [実演の難しさは何処にあると考えるか?]\mbox{}\\

新しい言葉を覚えても、その言葉を自由自在に使えるようになるまでには、一度フィジカルに落しこむ必要がある。そこがじゃないですか? 言わば「時間が掛る」んです。\\

--- {\bf フィジカルに落すときに、5拍4拍子に苦労してたように思うのだが?}\\

普段経験がないから。今回のことで解ったのが、3拍4連とか2拍3連も普通にやるんだけど、なんでそれが出来るかっていうと普段からそれをやってるから出来る。それは3拍4連を今回のような分解能で考えたことなんか一度もない。つまり、普通に聞いて覚えて弾いてを繰返してるから弾けてたわけで、普段から5拍4連の曲ばかり聞いて覚えてたら多分できるんだろう。

--- {\bf そういう意味ではシミュレーションしたデモ音源は役に立ちましたか?}\\

もちろん。フィジカルにインストールするという意味では意味では自分で練習用にカラオケを作ったので役にたったとは言えないけれど、デモ音源はそれが音楽に聞こえるのか、という問題のほうが大きいので「こういう風になるのか」という驚きのほうが大きい。あとは現場で処理する感じだった。\\

\item [今回の作品や手法の発展性についてどう考えるか?]\mbox{}\\

僕は今回の作品を見たときに一番最初に思い浮べたのがバルトークの弦のカルテットでしたね。これは対位法に当て嵌めるともっと面白いことができる。というかバルトークとかストラビンスキーはやってるんじゃないのかって思いました。でもどこでも応用可能じゃないですか? 自分の曲でもこういうことをやろうと思いました。だからそういう意味で分析してみようかなと思いました。縦のラインだけで追ってるとなにが起きてるのかが良くわからないのだけど、5個のフレーズとか4個のフレーズとかが同居して、そこで偶発的に起る和音が音楽のカラーリングになっていく考えかた音楽を作ってた、というか作れるじゃないですか? 調性決めておいたりスケール決めておいたりしておくと。それなら対位法的なそういうことに役に立つじゃないかなっていうのは思ったし、インプロの最中にやったらいいし、それは吉島さんが現場でやっている、というようなことでもあり、でもそれは急にやったらみんなビックリしちゃう。\\

--- {\bf トニー・ウイリアムスに対するマイルスの言及に似てる。}\\

それって、如何に理詰めで考えられた言葉でも人に伝わらないと意味がないていうか、いきなり訳のわからない言葉をしゃべりだしても周りがわからないとというところはあるので、それはやっぱり凄く難しいところだと思うんですけど、ただ、そういうことはやるべきだし、インプロでやるんじゃなくて楽曲の中でちゃんとやれば。\\

--- {\bf 構造としてしっかり持たせて?}\\

そう。\\

--- {\bf 今回の研究を問われたら新規性でもなんでもなくて「ああいう構造のものを記述して、音楽として成立させて、実演したところに価値がある」と思っているが。}\\

僕はこれは松尾芭蕉みたいなものだと思ってますよ。つまり「新しいボキャブラリーの研究」ですよ。「新しい表現方法の研究」ですよ。新規性とかそういう問題なくて、あたらしいのかもしれないのだけど。\\

--- {\bf ポピュラー音楽は同時時代的に起るからあんまり新規性を主張してもしょうがない。}

新しいフォーマットとまでいくとあれだけど、例えば5拍を4つに割るとなると、いままで16分音符では表現できなかったファジーな揺らぎが表現できるわけじゃないですか。それをベースにもっと違う表現のある、もっとシームレスな音楽が作れる可能性があるわけで。そのボキャブラリーの研究ってやっぱりもっと青でも赤でもなく紫色とか、紫色の中に黄色が混ざりましたみたいな、そういう表現の幅が広がるじゃないですか。それをもし自分のフィジカルの中にちゃんと持って自由に使えるようになれば。それは物凄い大きなこと。だから、松尾芭蕉みたいなもので、自分の言葉や感覚を五七五の中に当て嵌めるっていう、ベクトルは芭蕉は逆だけど、そういう研究をしていけば松尾芭蕉が文章を喋る必要はないわけですよ。そういう表現が可能になったわけだから。だから同じでもっと遂ぎ澄まされた研究をすれば、もっと表現の幅が広がっていく。みんなそう思ってくれると思うけど、現役ジャズ・ミュージシャンがそれをやる時間的な意味でどこまで価値を見いだせるかっていう、そこに尽きる。コストパフォーマンスの問題を取り払ったときにこんなに面白いことはないと思います。
\end{description}

\subsection{参加ミュージシャンへのインタビュー3 (金子 将昭、2017年12月19日)}

\begin{description}
\item[今回の作品の動画や音源を聴いての所感]\mbox{}\\

特に思うところはないが、普通に「できた」「ここまで作った」「凄いな」みたいな。演奏した側の「すごいな」という気持ちが。ゼロからイチを作ることが多いので、だからここまでやるのが如何に大変だよな、というのは凄く痛いほど解るんです。そういった点に関して、研究がどうこうというより「こういう結果を出した」ということが素晴しいな、と思う。しかもそこまでの細かいスケジューリングなどの道程を考えると重い。なにを思ってもなにを表現してもいいけど、そこまで行きつくのが凄い大変。研究の実験としての実演の評価自体は自分がどうこう言うことではないけれど、こういう理論よりの実験は役に立つ、立たないなどがほぼ解らないし、そういう部分が多いので評価としてはしようがない。逆にそういうところではなくて、作ったんだというところを評価してるし、そういうのって殆どいない。ジャズ・ミュージシャンしかり自分のCDは作っても、なにか思ったときに行動起せばいいが、実際にはやらない。そこを実際にやったところは凄いなと。\\

\item [作品の冊子ついて、内容が理解出来たか?]\mbox{}\\

凄く気合を入れて読みとけばなにをやりたいのかは伝わるんだろうけど、表面的に読んだだけど伝わらない。良く読みこめていないので自分的にもまだ入っていない。読みこんでみたときにどうかはまだ判断できない。\\

\item [冊子についての感想や、なにか自分に役に立つ知見のようなものはあったか?]\mbox{}\\

リズムに関していうと確かに考えることは多くて、人間のリズム認知については興味があるポイントでした。\\

\item [今までなにか類例となるような音楽を経験したことがあるか?]\mbox{}\\

もっとシンプルなものならあります。でも5拍と4拍のものはない。クロスにしてポリにはしてないけれど単順になんこ割りというのは自身の音楽体験でも練習したし、アドリブでも使ったりするけど、ポリまでは行かないという感じです。3とか4とかはあるけど、あそこまでやったことはない。でもポリとクロスを混ぜるようなことはない。\\

\item [普段やっていることと比較して差異はあるか?]\mbox{}\\

今回のクロスとポリの目的がどこにあるかしっかり理解しきれていないですが、自分が普段やっていることは平たく言うとビジネス的なアプローチでそこが違う。\\

--- {\bf そういう意味では今回の関係もビジネス的な楽曲の演奏を依頼して完遂するというビジネス的な関係でもありますね。その上でもやっぱり演奏者としての「ここまで行ければ完成」というプライド的なものも見えていたように思いますが?}\\

そこもあります。個人的に勝手に課しているところで多分自分が勝手に許せないところ。でもディレクターやプロデューサーが「OK」と言ったら、そこで止めざる得ないというあの場でもありました。あれはどうしてもある。でも関わるとそういう成長本能みたいなものがあるのかなと思います。\\

\item [実演の難しさは何処にあると考えるか?]\mbox{}\\

リズムが複合的になっているからでしょうね。単順にそこだと思う。簡単に言うとマルチタスクって難しいんだな、って思いました。人間が二つのことやると50\%ずつではやれず、30\%くらいになって効率が悪いということを聞いたことがあって。そういう大変さだと思います。ただ、人間が今回のようなものを処理するときにマルチタスクにやってない気がします。ドラマーの人がマルチで処理してないのと近いんじゃないか。なにか縦で捉えているとか。\\

ー {\bf 入り組んだ編み物のように一個の波のように捉えている?}\\

そう。多分、その捉えかたの学習が今迄ないので今回のようなものを演奏するのに困るんだと思う。それを学習していけば少しずつ慣れて出来る人は出来るんだと思う。人間が今回のようなものをどうやって処理してるのか? みたいなところには凄く興味がある。そう言う視点だと演奏するという行為がとても知的というか芸術的な作業だと思う。\\

\item [今回の作品や手法の発展性についてどう考えるか?]\mbox{}\\

世の中の人がこういう曲をどういう風に使えるかを考えられるのではないか。レコーディングの時にも今回の音楽は「本屋で流れるといい」と言ったが、自分は「音楽が場所に同化する」と考えているので、場所に同化してクリエイティブに作れるといい。\\

ー {\bf この音楽に相応わしくなる場所があるはずというのをこの音楽の発展の先に見ている?}\\

そうです。それはこの音楽に限らず芸術にしてもなんにしても人に求められないと存在出来ずに消えていくだけなので、生れたものに対していい場所がどこかにあるはずなんです。音楽はBGMの流れなら場所に同化していくと考えるので、アーティストはその音楽がどこに必要なのかを考えることが大事だと思います。
\end{description}

\subsection{他ジャンルのミュージシャンへのインタビュー (安藤 真也、原口 順、2017年12月13日)}

\begin{description}
\item[今回の作品の動画や音源を聴いての所感]\mbox{}\\

{\bf 安藤} Etude #2でベースの人が口で5を数えてるのを見て「あれ、5なんだ」ってのが始めのとっかかりでした。そこからどっちもどうとてもで取れるんだ、っていうことが解った。曲としてはすごく良かったです。リズム割りが独特で変なので、独特のグルーブ感があって凄く面白かった。カッコ良かった。メロディ楽器の人が、変拍子のをむりくり入れていってずっとそれをキープしながらソロを弾くのを自分もやるんですけど、やっぱり難しい。あれはまた未来に向けてですよね。みんなが変拍子がまず最初にスタンダードに5拍子の曲が普通に一杯あって、そこで5になれててという時代がきたら、違うリズムでバッと入れていけることが段々可能になるのではないかな、と思いましたね。ソロとかの部分は。\\

{\bf 原口} Etude #2はドラムが4で取れる感じと5で取れる感じとで、自分の聴きかたを変えるとどちらにも聞こえやすくて、それでミュージシャンが側としてはそれがあることで演奏しやすいのかな、と思って聞いてました。Etude #2のベースとピアノのリズムの乗り換えの部分が曲としていい。音楽としてグっと軸が変る感じが面白い。そこでベースが4分音符4つを打つというのが良かった。メロディー的に色々やるよりここはこういう4分音符のラインが、変化を強調するので変った感がグッと来て面白かったです。こういった譜面とか説明とかなにも聞かずにただ音だけ聞いたらなにをやってるのかはそれだけじゃわからない感じで、分析/解析は出来ないけど曲としては良かった。\\

\item [作品の冊子ついて、内容が理解出来たか?]\mbox{}\\

{\bf 原口} 内容は解ったけども、イメージを掴む為に聞いて、納得、という感じ。読んで、冊子を見て、そのあとで聴きながら譜面を見てて「ああそうかそうか」と納得できるという感じでした。\\

{\bf 安藤} (事前に)作品の作者と話をしてるので、その先入観があるので、まったく知らないで、これをパッとこれを見たときに理解できるかはわからない。\\

\item [冊子についての感想や、なにか自分に役に立つ知見のようなものはあったか?]\mbox{}\\

{\bf 原口} サントゥールやタブラのような両手が使える楽器だから可能なのですが、ここでもドラムがやってるような同時に左右の手で別の拍を演奏するというのではなく、ふたりの奏者がそれぞれ違うリズムで演奏する例は色々な例があるのだけど、一人の奏者が両手で別の拍をやるというアプローチは過去にはなかったです。安藤さんはそれを大和さんと会ってから、前からモヤモヤと考えていたことが形になったとは夏の時に言っていて、それを最近実際にやっています。\\

{\bf 安藤} どうしたらいいのかな、とは思いつつまだ基本的な状態なので、普通に右手が7で左手が4というだけの基本的なことしかしてないんですけれど。元が10拍子の時に、真ん中で割ってその5拍をに対して7と4を入れています。\\

{\bf 原口} そのときにタブラが5をキープしている中で、7のフレーズを片手でやって片手が4をやる、というのをメロディーでやってるので、それが面白いメロディで聞こえるというアプローチを最近やっています。\\

--- {\bf その10を5で割るときには所謂ある塊としての小節が共有された状態でポリリズムをやっているということで、アフロポリみたいなものと似てる考えかたですよね?}

{\bf 安藤} 2拍3連みたいな考え方ですね。

\item [今までなにか類例となるような音楽を経験したことがあるか?]\mbox{}\\

{\bf 原口} ザキール・フセインとミッキー・ハートが76年に演奏してる例ですが、すごく大きなスパンで取るという例ですが、この類例にあたるわけじゃないけれど、曲の解説を読むと102.5ビートの中で一つのメロディーを作る例があります。ただ単にこの曲聞いても102.5のメロディーになってる風に取れないんですけど説明を読むとそうなってるそうです。これはインド古典というわけですがザキール・フセインというタブラ奏者がやっています。いろんなパーカッションを入れて演奏しています。\\

{\bf 安藤} ザキール・フセインのやっていることは、アレンジとか楽曲を変えたりはしてるんですけど、考えかたはインド音楽のことしかやっていないです。シャクティとかでもいろいろ変えてるけれどインド古典しかやっていない。\\

--- {\bf これ聞いてみても一拍の長さがどの単位になってるか解らないですよね?}

{\bf 安藤} 解らないですよね。\\

\item [実演の難しさは何処にあると考えるか?]\mbox{}\\

{\bf 安藤} 難しさは、冊子内の表みたいなものを出してしまえば難しさはないんですけど、それを自分の弾きたい速度までやりつづけるのが長いですよね。簡単なワンフレーズ、これちょっと難しいなぁっていうのを練習するどころの話じゃないんですよね。辿りつくのかっていうくらい長い間かかりますね、一人で自分がそれをやろうとすると。集団でやるようになるのはもう少し簡単で、そのハメ方を覚えていけば、もう自分の中にそれがポンと入れば。その変り自分が出してる持ち駒の中に5拍子の中にはこうなるよねとか、あと7拍子はこうだよねとか、3はこうだよね、6、12はこうだよね、ていうのが自分の中にあればそこにぼんぼんハメていけるんですけど、一人がこのドラムの人みたいにバラバラのことし始めるのは時間が掛りますね。\\

--- {\bf そういう演奏と実際アンサンブルをやってるときには、自分がつられてしまうとかそういうことって起ったりするのかなって想像するんですが。}\\

{\bf 安藤} つられますよ。\\

{\bf 原口} つられそうになりますよ。\\

--- {\bf それってどうやってつられないように保っているんですか?}\\

{\bf 安藤} 自分がメトロノームになった気持ち。\\

{\bf 原口} リズムを自分の中で歌っているんですけど、基本となるリズムを心の中で。歌いながら、こっちを聞いてるけど。頭を見てる、そのメロディの打点の。\\

{\bf 安藤} 打点を見てるのと、一応違うのでこう周ってるというイメージは持ってますね。\\

--- {\bf アンサンブルでどこまで相手のことを聞いてるんだろう、という点はどうでしょうか? 時々に必要なところにフォーカスしていて、アンサブル全体を聞いてるという状態は起り難いと思うのですがお二人はどうなんですか?}\\

{\bf 原口} ドラムの方は聞いてる、というのと同じような感じは役割としてある。旋律を歌うほうは目印となる音は聞いてて。\\

{\bf 安藤} ポリリズムの時は、例えば10拍子だと頭と6拍目と3拍目の裏、3拍目も気にしてるかな。その辺は気にしてるんですけど、あとはあまり気にしてないっていうか。\\

ーーー {\bf リズムの構造的な理解が相互にあってアンサンブルをやっている感じなのですかね?}\\

{\bf 安藤} そうですね。この手法自体がリズムじゃないですか? リズムだけでも成り立つものなので、そこにメロディを載せていってるということなので、やっぱりリズムはすごく、インドは特にリズムの意識が高いですよね。そうじゃないミュージシャンって凄く多いと思うんですけど、インド人の演奏を聞いてるとリズムの意識が高いですよね。1拍に3を入れていくっていうやり方とかもあるんですけど、これを5でズレていくとか。\\

{\bf 原口} どこに自分がいるのかっていうか、拍の空白の中に。拍の1/3のところとか1/5のところとか。それを意識しろ、というか、するようにしていますね。\\

{\bf 安藤} ポリリズム的なことをやっている時はすごく相手を聞いてる。いつものときよりポリズム的なトリック的なことをやってるときは打点があってないとダメなので相当聞いてる。元のビートをやってる側は、ひっぱりこまれちゃうので、相手を聴きながらでも、もうメトロノームの様にやるんですけど、アプローチしていってる側のほうがどちらかというと聞いてる。聴きながらフレーズを収めていく。\\

--- {\bf 今のお話を聞いてるとハーモニックリズムというよりも本当に「リズム」ていうもので自分の位置がどこにあるのかを自分が捉えられるようになっている、という状態なんですね?}\\

{\bf 原口} そうですね。\\

{\bf 安藤} 勿論、そのハーモニックリズムというものを使ったりしますけど、その前にリズムっていうものを(自分の先生に)言われます。ラヴィ・シャンカールの先生の息子がいるんですけど、その人は口が何拍子で、手が何拍子で、なんか5つか6つ、口と両手、両足がバラバラっていう。\\

{\bf 原口} (自分も)3つまでは出来るんですよ。口と両手までは別のことが出来るんですけど。両足となると(難しい)。なんかズレ込んでいくという感覚でインド古典音楽の特徴的なのは、この纏まりを跨いで塊のメロディーラインがズレていくとか、、、\\

{\bf 安藤} 明らかに10拍子の曲に7でいってっていうことはある。\\

--- {\bf 10拍子の中に拍を共有したまま7拍を共有していてそれがズレていく。}\\

{\bf 安藤} テハイみたいな感じでズレていくというのもあります。\\

--- {\bf その時に縦線のビートはそろっている。}\\

{\bf 安藤} はい。それもあります。\\

\item [今回の作品や手法の発展性についてどう考えるか?]\mbox{}\\

{\bf 原口} 私は具体的な感じで、自分の演奏にもこれが取り入れられる要素があるなって思いました。\\

{\bf 安藤} それは勿論あって、でこういったリズムが多重になってる音楽っていうのが、普通に普通に巷を歩いてて掛っている時代が来て欲しいですよね。\\

{\bf 原口} すごく単順な音楽が蔓延してる世の中だから、なんかつまんないっていうか、それが一見聞いたら誰もがすぐわかるようなものばかり流れているけど、一見聞いて分析がちゃんとこうじっくり向きあわないと解らないじゃないですか、これとか。どうなってるかわからないけど、音楽として聞いたらすごい美しいとかそういう音楽がもっと広がってほしい。\\

{\bf 安藤} なんで4なのか忘れちゃったんですけど、ようするにこういう芸術音楽とされるものがないと、民衆が聴く側が強い状態、経済的なものでも昔はパトロンが居て楽師を雇っていうので、音楽が伸びてきたと思うんですよね。それがスポンサーが出したりとか、それが今わりとチケットを売ってそこからっいう時代の聴く側が強いっていう構図じゃないですか? なんかで読んだのは民衆が選ぶようになっていくと音楽とかっていうものはどんどん低能化していくっていうのは聞いたことがあって、やっぱりツアーを周っていたりすると実際にそうなんですよね。もっと面白いこともあるんだよ、っていうことをやらないっていうか。感じたままっていうのもいいんだけど小さな自分のスキルの中で納得できるものじゃないと認めないっていう感じの人が多い。そういう風に段々なっていくとつまんなくなるじゃないですか、世の中が。だからこういう音楽がもっと広がっていく未来を想像してますよね。友達の店で掛ってたりしてほしい。
\end{description}

% リファレンス
\bibliographystyle{chicago}
\begin{thebibliography}{99}
\bibitem[村井 2017]{murai:01} 村井 康司 2017 『あなたの聴き方を変えるジャズ史』株式会社シンコーミュージック・エンターテイメント
\bibitem[ハラリ 2016]{harari:01}ユヴァル・ノア・ハラリ(柴田裕之 訳) 2016 『サピエンス全史 上下合本版 文明の構造と人類の幸福』 (Kindle Locations 3000-3010) 河出書房新社
\bibitem[ファン=デル=マーヴェ 1999]{merwe:01} ピーター・ファン=デル=マーヴェ(中村 とうよう 訳) 1999 『ポピュラー音楽の基礎理論』株式会社ミュージック・マガジン
\bibitem[菊地・大谷 2004]{kikuchi-ootani:01} 菊地 成孔, 大谷 能生 2004 『憂鬱と官能を教えた学校』河出書房新社
\bibitem[淺香 1977]{asaka:01} 淺香 淳 1977 『新音楽辞典』音楽之友社
\bibitem[依田・小野 2016]{yoda-ono:01} 依田 翔、小野 貴史 2016 『ポリリズム類型における楽理的分析』信州大学教育学部研究論集 第9号 pp.169-188
\bibitem[ボスール 2008]{bosseur:01} ジャン=イヴ・ボスール(栗原 詩子 訳) 2008 『現代音楽を読み解く88のキーワード』音楽之友社
\bibitem[塚田 2016]{tsukada:01} 塚田 健一 2016 『アフリカ音楽の正体』音楽之友社
\bibitem[デーヴァ 1994]{deva:01} B・C・デーヴァ(中川 博志 訳) 1994 『インド音楽序説』東方出版
\bibitem[ザックス 1979]{sachs:01} クルト・ザックス(岸辺 成雄 訳) 1979 『リズムとテンポ』音楽之友社
\bibitem[マハディ 1998]{mahdi:01} サラーフ・アル・マハディ(松田 嘉子 訳) 1998 『アラブ音楽』PASTORALE出版
\bibitem[Clave 2008]{clave:01} Rumba Clave. 2008 ''Rumba Clave: An Illustrated Analysis''\\
\url{http://rumbaclave.blogspot.jp/2007/08/rumba-clave-illustrated-analysis.html}:2018年1月3日(水)アクセス
\bibitem[オブリスト 2015]{obrist:01} ハンス・ウルリッヒ・オブリスト(篠儀 直子, 内山 史子, 西原 尚 訳) 2015 『ミュージック 「現代音楽」を作った作曲家たち』フィルムアート社
\bibitem[ストリックランド 1998]{strickland:01} E・ストリックランド(柿沼 敏江, 米田 栄 訳) 1998 『アメリカン・ニュー・ミュージック』勁草書房
\bibitem[菊地 2003]{kikuchi:01} 菊地 成孔 2003 『菊地成孔氏自身によるニュー・アルバム 『Structure et Force』の全曲解説を公開!』 \url{https://web.archive.org/web/20031002092226/http://www.bls-act.co.jp/artists/dcprg.html}:2018年1月8日(月)アクセス
\bibitem[由比 1996]{yubi:01} 由比 邦子 1996 『ポピュラー・リズムのすべて』勁草書房
\end{thebibliography}

\begin{referenceworks}{99}
\bibitem[大和 2017a]{yamato:01} 大和 比呂志 2017a ''Music for multilayered temporal structures'' \url{https://goo.gl/o1pTEk}
\bibitem[大和 2017b]{yamato:02} 大和 比呂志 2017b ''4by5'' \url{https://github.com/dropcontrol/4by5}
\bibitem[大和 2017c]{yamato:03} 大和 比呂志 2017c ''3by4by5'' \url{https://github.com/dropcontrol/3by4by5}
\bibitem[大和 2017d]{yamato:04} 大和 比呂志 2017d 記録映像''Music for multilayered temporal structures'' \url{https://youtu.be/3AuNz4GiLsI}
\bibitem[大和 2017e]{yamato:06} 大和 比呂志 2017e ''Test6 (Piano Etude for D Dorian)'' \url{https://soundcloud.com/dropcontrol/test6piano-ensemble-for-d-dorian}.
\bibitem[大和 2017f]{yamato:07} 大和 比呂志 2017f ''Test5 (Percussion Ensemble)'' \url{https://soundcloud.com/dropcontrol/test5-percussion-ensemble}.
\bibitem[大和 2016]{yamato:05} 大和 比呂志 2016 {\it New York 2997} ''PCM15 BEAT MUSICS/ペンギン音楽大学院2015年度ブラックミュージッククラス卒業制作集'' Sony Music Artists Inc.
\bibitem[Davis 1959]{miles:01} Davis, Miles 1959 {\it So What} ''Kind of Blue'' Columbia
\bibitem[Davis 1970]{miles:02} Davis, Miles 1970 {\it Bitches Brew} Columbia
\bibitem[Davis 1956]{miles:03} Davis, Miles 1956 {\it Bye Bye Blackbird} ''Round About Midnight''  Columbia
\bibitem[Reich 2013]{reich:01} Reich, Steve 2013 {\it Piano Phase} ''Steve Reich Early Works'' ワーナーミュージック・ジャパン
\bibitem[Reich 2016]{reich:02} Reich, Steve 2016 {\it Violin Phase} ''The Ecm Recordings'' Ecm Records
\bibitem[Reich 1995]{reich:03} Reich, Steve 1995 ''Drumming'' Nonesuch
\bibitem[Date Course Pentagon Royal Garden 2001]{works-kikuchi:01} Date Course Pentagon Royal Garden 2001 {\it PLAYMATE AT HANOI} ''REPORT FROM IRON MOUNTAIN'' Pヴァインレコード
\bibitem[Date Course Pentagon Royal Garden 2003]{works-kikuchi:02} Date Course Pentagon Royal Garden 2003 {\it structure I la structure de la magie monderne /構造 1 (現代呪術の構造)} ''Structure et Force'' Pヴァインレコード
\bibitem[Glasper 2012]{Glasper:02} Glasper, Robert 2012 ''Black Radio'' Blue Note Records
\end{referenceworks}

%改ページや図版を章に収めたい場合は以下のように改ページすると収まる。
\afterpage{\clearpage}
\newpage

\section*{謝辞}

本研究に進めるにあたって、次の方々に大変にお世話になりました。まず、三輪 眞弘教授、前田 真二朗教授、小林 昌廣教授には2年に渡り、様々なご助言やご指導を頂いけたことに深く感謝をいたします。特に三輪 眞弘教授には「この後の人生で音楽にどう向きあっていくか」ということを深く考える機会を与えて頂けたことに深く感謝いたします。また、修士作品に参加してくれたジャズ・ミュージシャンの吉島 智仁氏、川本 悠自氏、金子 将昭氏、エンジニアの速水 直樹氏、記録映像の撮影を行ってくれた大石 桂誉氏、ミックス及びマスタリングを行ってくれた大久保 雅基氏に感謝いたします。特に大石 桂誉氏には、研究を進めるに過程でも様々な助言や助力を頂きました。\\

ATPの安藤 泰彦教授、NxPCの仲間達、特に平林 真実教授と永松 歩氏に感謝いたします。お陰で今迄行えなかった活動を幾つも実現することができました。また、批評家として様々な角度から助言を頂いた松井 茂准教授にも感謝いたします。松井 茂准教授から昨年夏にメールで頂いた「大和さんにとっての名曲とは?」という質問にはまだ答えていないままなのですが、答は用意してありますから次の機会にまた議論できたら嬉しいです。\\

IAMASに入学するにあたって僕の背中を押してくれた、山辺 真幸氏、古賀 早氏、平田 大治氏、山田 洋氏に感謝いたします。特に山辺、古賀、平田の3名はこの10年以上に渡って、常に僕の友人でありメンターでもありました。またアライアンス・ポートの当時の仲間と、そこに関わってくれた全てのみなさんにも同じく感謝を送ります。\\

この論文は小川 銀次氏、菊地 成孔氏の両氏に捧げます。僕は銀次さんの不詳の弟子でしたが、ギター弾くという事について様々なことを教わりました。また菊地さんの私塾である「ペンギン音楽大学」とその仲間達にも感謝します。特に金山 弘平氏と本條 晴一郎氏に。彼等と共に菊地さんの元で学んだことがこの研究の全ての始まりでした。\\

常に僕を応援してくれている父、母、弟に感謝の意を捧げます。いつか母が言った「あなたが音楽やってて本当に良かった。やってなかったらとっくに死んじゃってたと思うの」という言葉が、今の自分を端的に表わしていると心から思います。\\

最後に。この論文がそうであるように音楽は常に誰かとの関係でなりたつ芸術で「一人では出来ないこと」だと思います。ですから、ここに書き切れない沢山の方がいる、ということを最後に記してその方々への謝辞に代えさせていただきます。\\

皆様、本当にどうもありがとうございました。

\end{document}
